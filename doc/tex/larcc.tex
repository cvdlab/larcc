% -------------------------------------------------------------------------
% ------ nuweb macros (redefine as desired, or omit with "nuweb -p") ------
% -------------------------------------------------------------------------
\providecommand{\NWtxtMacroDefBy}{Macro defined by}
\providecommand{\NWtxtMacroRefIn}{Macro referenced in}
\providecommand{\NWtxtMacroNoRef}{Macro never referenced}
\providecommand{\NWtxtDefBy}{Defined by}
\providecommand{\NWtxtRefIn}{Referenced in}
\providecommand{\NWtxtNoRef}{Not referenced}
\providecommand{\NWtxtFileDefBy}{File defined by}
\providecommand{\NWsep}{${\diamond}$}
\providecommand{\NWlink}[2]{\hyperlink{#1}{#2}}
\providecommand{\NWtarget}[2]{% move baseline up by \baselineskip 
  \raisebox{\baselineskip}[1.5ex][0ex]{%
    \mbox{%
      \hypertarget{#1}{%
        \raisebox{-1\baselineskip}[0ex][0ex]{%
          \mbox{#2}%
}}}}}
% -------------------------------------------------------------------------

\documentclass[11pt,oneside]{article}	%use"amsart"insteadof"article"forAMSLaTeXformat
\usepackage{geometry}		%Seegeometry.pdftolearnthelayoutoptions.Therearelots.
\geometry{letterpaper}		%...ora4paperora5paperor...
%\geometry{landscape}		%Activateforforrotatedpagegeometry
%\usepackage[parfill]{parskip}		%Activatetobeginparagraphswithanemptylineratherthananindent
\usepackage{graphicx}				%Usepdf,png,jpg,orepsßwithpdflatex;useepsinDVImode
								%TeXwillautomaticallyconverteps-->pdfinpdflatex		
\usepackage{amssymb}
\usepackage{amsmath}
\usepackage{amsthm}
\newtheorem{definition}{Definition}
\newtheorem{theorem}{Theorem}
\usepackage[colorlinks]{hyperref}

%----macros begin---------------------------------------------------------------
\usepackage{color}
\usepackage{amsthm}

\def\conv{\mbox{\textrm{conv}\,}}
\def\aff{\mbox{\textrm{aff}\,}}
\def\E{\mathbb{E}}
\def\R{\mathbb{R}}
\def\Z{\mathbb{Z}}
\def\tex{\TeX}
\def\latex{\LaTeX}
\def\v#1{{\bf #1}}
\def\p#1{{\bf #1}}
\def\T#1{{\bf #1}}

\def\vet#1{{\left(\begin{array}{cccccccccccccccccccc}#1\end{array}\right)}}
\def\mat#1{{\left(\begin{array}{cccccccccccccccccccc}#1\end{array}\right)}}

\def\lin{\mbox{\rm lin}\,}
\def\aff{\mbox{\rm aff}\,}
\def\pos{\mbox{\rm pos}\,}
\def\cone{\mbox{\rm cone}\,}
\def\conv{\mbox{\rm conv}\,}
\newcommand{\homog}[0]{\mbox{\rm homog}\,}
\newcommand{\relint}[0]{\mbox{\rm relint}\,}

%----macros end-----------------------------------------------------------------

\title{The basic \texttt{larcc} module
\footnote{This document is part of the \emph{Linear Algebraic Representation with CoChains} (LAR-CC) framework~\cite{cclar-proj:2013:00}. \today}
}
\author{The LARCC team}
%\date{}							%Activatetodisplayagivendateornodate

\begin{document}
\maketitle
\nonstopmode

\tableofcontents
\newpage


\section{Basic representations}

A few basic representation of topology are used in LARCC. They include some common sparse matrix representations: CSR (Compressed Sparse Row),  CSC (Compressed Sparse Column),   COO (Coordinate Representation), and BRC (Binary Row Compressed). 

\subsection{BRC (Binary Row Compressed)}

We denote as BRC (Binary Row Compressed) the standard input representation of our LARCC framework. A BRC representation is an array of arrays of integers, with no requirement of equal length for the component arrays. The BRC format is used to represent a (normally sparse) binary matrix. Each component array corresponds to a matrix row, and contains the indices of columns that store a 1 value. No storage is used for 0 values.

\paragraph{BRC format example}

Let $A = (a_{i,j} \in \{0,1\})$ be a binary matrix. The notation $\texttt{BRC}(A)$ is used for the corresponding data structure.
\[
A = \mat{
0,1,0,0,0,0,0,1,0,0\\
0,0,1,0,0,0,0,0,0,0\\
1,0,0,1,0,0,0,0,0,1\\
1,0,0,0,0,0,1,0,0,0\\
0,0,0,0,0,1,1,1,0,0\\
0,0,1,0,1,0,0,0,1,0\\
0,0,0,0,0,0,0,0,0,0\\
0,1,0,0,0,0,0,1,0,1\\
0,0,0,1,0,0,0,0,1,0\\
0,1,1,0,1,0,0,0,0,0\\
}
\qquad\mapsto\qquad \texttt{BRC}(A) =
\begin{minipage}[c]{5cm}
\begin{verbatim}
[[1,7],
 [2],
 [0,3,9],
 [0,6],
 [5,6,7],
 [2,4,8],
 [],
 [1,7,9],
 [3,8],
 [1,2,4]]
\end{verbatim}
\end{minipage}
\]


\subsection{Format conversions}

First we give the function \texttt{format} to make the transformation from the sparse matrix as a list of triples \emph{(row,column,value)} for each non-zero element, to the \texttt{scipy.sparse} format corresponding to the \texttt{shape} parameter, set by default to \texttt{"csr"}, that stands for \emph{Compressed Sparse Row}, the normal matrix format of the LARCC framework. 
%-------------------------------------------------------------------------------
\begin{flushleft} \small
\begin{minipage}{\linewidth} \label{scrap1}
\protect\makebox[0ex][r]{\NWtarget{nuweb3a}{\rule{0ex}{0ex}}\hspace{1em}}$\langle\,$From list of triples to scipy.sparse\nobreak\ {\footnotesize 3a}$\,\rangle\equiv$
\vspace{-1ex}
\begin{list}{}{} \item
\mbox{}\verb@def format(triples,shape="csr"):@\\
\mbox{}\verb@    n = len(triples)@\\
\mbox{}\verb@    data = arange(n)@\\
\mbox{}\verb@    ij = arange(2*n).reshape(2,n)@\\
\mbox{}\verb@    for k,item in enumerate(triples):@\\
\mbox{}\verb@        ij[0][k],ij[1][k],data[k] = item@\\
\mbox{}\verb@    return scipy.sparse.coo_matrix((data, ij)).asformat(shape)@\\
\mbox{}\verb@@{\NWsep}
\end{list}
\vspace{-1ex}
\footnotesize\addtolength{\baselineskip}{-1ex}
\begin{list}{}{\setlength{\itemsep}{-\parsep}\setlength{\itemindent}{-\leftmargin}}
\item \NWtxtMacroRefIn\ \NWlink{nuweb16a}{16a}.
\end{list}
\end{minipage}\\[4ex]
\end{flushleft}
%-------------------------------------------------------------------------------
%-------------------------------------------------------------------------------
\begin{flushleft} \small
\begin{minipage}{\linewidth} \label{scrap2}
\protect\makebox[0ex][r]{\NWtarget{nuweb3b}{\rule{0ex}{0ex}}\hspace{1em}}$\langle\,$Brc to Coo transformation\nobreak\ {\footnotesize 3b}$\,\rangle\equiv$
\vspace{-1ex}
\begin{list}{}{} \item
\mbox{}\verb@def cooCreateFromBrc(ListOfListOfInt):@\\
\mbox{}\verb@    COOm = [[k,col,1] for k,row in enumerate(ListOfListOfInt)@\\
\mbox{}\verb@            for col in row ]@\\
\mbox{}\verb@    return COOm@\\
\mbox{}\verb@@{\NWsep}
\end{list}
\vspace{-1ex}
\footnotesize\addtolength{\baselineskip}{-1ex}
\begin{list}{}{\setlength{\itemsep}{-\parsep}\setlength{\itemindent}{-\leftmargin}}
\item \NWtxtMacroRefIn\ \NWlink{nuweb16a}{16a}.
\end{list}
\end{minipage}\\[4ex]
\end{flushleft}
%-------------------------------------------------------------------------------
%-------------------------------------------------------------------------------
\begin{flushleft} \small
\begin{minipage}{\linewidth} \label{scrap3}
\protect\makebox[0ex][r]{\NWtarget{nuweb3c}{\rule{0ex}{0ex}}\hspace{1em}}$\langle\,$Test example of Brc to Coo transformation\nobreak\ {\footnotesize 3c}$\,\rangle\equiv$
\vspace{-1ex}
\begin{list}{}{} \item
\mbox{}\verb@print "\n>>> cooCreateFromBrc"@\\
\mbox{}\verb@V = [[0, 0], [1, 0], [2, 0], [0, 1], [1, 1], [2, 1]]@\\
\mbox{}\verb@FV = [[0, 1, 3], [1, 2, 4], [1, 3, 4], [2, 4, 5]]@\\
\mbox{}\verb@EV = [[0,1],[0,3],[1,2],[1,3],[1,4],[2,4],[2,5],[3,4],[4,5]]@\\
\mbox{}\verb@cooFV = cooCreateFromBrc(FV)@\\
\mbox{}\verb@cooEV = cooCreateFromBrc(EV)@\\
\mbox{}\verb@print "\ncooCreateFromBrc(FV) =\n", cooFV@\\
\mbox{}\verb@print "\ncooCreateFromBrc(EV) =\n", cooEV@\\
\mbox{}\verb@@{\NWsep}
\end{list}
\vspace{-1ex}
\footnotesize\addtolength{\baselineskip}{-1ex}
\begin{list}{}{\setlength{\itemsep}{-\parsep}\setlength{\itemindent}{-\leftmargin}}
\item \NWtxtMacroRefIn\ \NWlink{nuweb16b}{16b}.
\end{list}
\end{minipage}\\[4ex]
\end{flushleft}
%-------------------------------------------------------------------------------
%-------------------------------------------------------------------------------
\begin{flushleft} \small
\begin{minipage}{\linewidth} \label{scrap4}
\protect\makebox[0ex][r]{\NWtarget{nuweb3d}{\rule{0ex}{0ex}}\hspace{1em}}$\langle\,$Coo to Csr transformation\nobreak\ {\footnotesize 3d}$\,\rangle\equiv$
\vspace{-1ex}
\begin{list}{}{} \item
\mbox{}\verb@def csrCreateFromCoo(COOm):@\\
\mbox{}\verb@    CSRm = format(COOm,"csr")@\\
\mbox{}\verb@    return CSRm@\\
\mbox{}\verb@@{\NWsep}
\end{list}
\vspace{-1ex}
\footnotesize\addtolength{\baselineskip}{-1ex}
\begin{list}{}{\setlength{\itemsep}{-\parsep}\setlength{\itemindent}{-\leftmargin}}
\item \NWtxtMacroRefIn\ \NWlink{nuweb16a}{16a}.
\end{list}
\end{minipage}\\[4ex]
\end{flushleft}
%-------------------------------------------------------------------------------
%-------------------------------------------------------------------------------
\begin{flushleft} \small
\begin{minipage}{\linewidth} \label{scrap5}
\protect\makebox[0ex][r]{\NWtarget{nuweb4a}{\rule{0ex}{0ex}}\hspace{1em}}$\langle\,$Test example of Coo to Csr transformation\nobreak\ {\footnotesize 4a}$\,\rangle\equiv$
\vspace{-1ex}
\begin{list}{}{} \item
\mbox{}\verb@print "\n>>> csrCreateFromCoo"@\\
\mbox{}\verb@csrFV = csrCreateFromCoo(cooFV)@\\
\mbox{}\verb@csrEV = csrCreateFromCoo(cooEV)@\\
\mbox{}\verb@print "\ncsr(FV) =\n", repr(csrFV)@\\
\mbox{}\verb@print "\ncsr(EV) =\n", repr(csrEV)@\\
\mbox{}\verb@@{\NWsep}
\end{list}
\vspace{-1ex}
\footnotesize\addtolength{\baselineskip}{-1ex}
\begin{list}{}{\setlength{\itemsep}{-\parsep}\setlength{\itemindent}{-\leftmargin}}
\item \NWtxtMacroRefIn\ \NWlink{nuweb16b}{16b}.
\end{list}
\end{minipage}\\[4ex]
\end{flushleft}
%-------------------------------------------------------------------------------
%-------------------------------------------------------------------------------
\begin{flushleft} \small
\begin{minipage}{\linewidth} \label{scrap6}
\protect\makebox[0ex][r]{\NWtarget{nuweb4b}{\rule{0ex}{0ex}}\hspace{1em}}$\langle\,$Brc to Csr transformation\nobreak\ {\footnotesize 4b}$\,\rangle\equiv$
\vspace{-1ex}
\begin{list}{}{} \item
\mbox{}\verb@def csrCreate(BRCm,shape=(0,0)):@\\
\mbox{}\verb@    if shape == (0,0):@\\
\mbox{}\verb@        out = csrCreateFromCoo(cooCreateFromBrc(BRCm))@\\
\mbox{}\verb@        return out@\\
\mbox{}\verb@    else:@\\
\mbox{}\verb@        CSRm = scipy.sparse.csr_matrix(shape)@\\
\mbox{}\verb@        for i,j,v in cooCreateFromBrc(BRCm):@\\
\mbox{}\verb@            CSRm[i,j] = v@\\
\mbox{}\verb@        return CSRm@\\
\mbox{}\verb@@{\NWsep}
\end{list}
\vspace{-1ex}
\footnotesize\addtolength{\baselineskip}{-1ex}
\begin{list}{}{\setlength{\itemsep}{-\parsep}\setlength{\itemindent}{-\leftmargin}}
\item \NWtxtMacroRefIn\ \NWlink{nuweb16a}{16a}.
\end{list}
\end{minipage}\\[4ex]
\end{flushleft}
%-------------------------------------------------------------------------------
%-------------------------------------------------------------------------------
\begin{flushleft} \small
\begin{minipage}{\linewidth} \label{scrap7}
\protect\makebox[0ex][r]{\NWtarget{nuweb4c}{\rule{0ex}{0ex}}\hspace{1em}}$\langle\,$Test example of Brc to Csr transformation\nobreak\ {\footnotesize 4c}$\,\rangle\equiv$
\vspace{-1ex}
\begin{list}{}{} \item
\mbox{}\verb@print "\n>>> csrCreateFromCoo"@\\
\mbox{}\verb@V = [[0, 0], [1, 0], [2, 0], [0, 1], [1, 1], [2, 1]]@\\
\mbox{}\verb@FV = [[0, 1, 3], [1, 2, 4], [1, 3, 4], [2, 4, 5]]@\\
\mbox{}\verb@csrFV = csrCreate(FV)@\\
\mbox{}\verb@print "\ncsrCreate(FV) =\n", csrFV@\\
\mbox{}\verb@@{\NWsep}
\end{list}
\vspace{-1ex}
\footnotesize\addtolength{\baselineskip}{-1ex}
\begin{list}{}{\setlength{\itemsep}{-\parsep}\setlength{\itemindent}{-\leftmargin}}
\item \NWtxtMacroRefIn\ \NWlink{nuweb16b}{16b}.
\end{list}
\end{minipage}\\[4ex]
\end{flushleft}
%-------------------------------------------------------------------------------

\section{Matrix operations}

%-------------------------------------------------------------------------------
\begin{flushleft} \small
\begin{minipage}{\linewidth} \label{scrap8}
\protect\makebox[0ex][r]{\NWtarget{nuweb4d}{\rule{0ex}{0ex}}\hspace{1em}}$\langle\,$Query Matrix shape\nobreak\ {\footnotesize 4d}$\,\rangle\equiv$
\vspace{-1ex}
\begin{list}{}{} \item
\mbox{}\verb@def csrGetNumberOfRows(CSRm):@\\
\mbox{}\verb@    Int = CSRm.shape[0]@\\
\mbox{}\verb@    return Int@\\
\mbox{}\verb@    @\\
\mbox{}\verb@def csrGetNumberOfColumns(CSRm):@\\
\mbox{}\verb@    Int = CSRm.shape[1]@\\
\mbox{}\verb@    return Int@\\
\mbox{}\verb@@{\NWsep}
\end{list}
\vspace{-1ex}
\footnotesize\addtolength{\baselineskip}{-1ex}
\begin{list}{}{\setlength{\itemsep}{-\parsep}\setlength{\itemindent}{-\leftmargin}}
\item \NWtxtMacroRefIn\ \NWlink{nuweb16a}{16a}.
\end{list}
\end{minipage}\\[4ex]
\end{flushleft}
%-------------------------------------------------------------------------------
%-------------------------------------------------------------------------------
\begin{flushleft} \small
\begin{minipage}{\linewidth} \label{scrap9}
\protect\makebox[0ex][r]{\NWtarget{nuweb5a}{\rule{0ex}{0ex}}\hspace{1em}}$\langle\,$Test examples of Query Matrix shape\nobreak\ {\footnotesize 5a}$\,\rangle\equiv$
\vspace{-1ex}
\begin{list}{}{} \item
\mbox{}\verb@print "\n>>> csrGetNumberOfRows"@\\
\mbox{}\verb@print "\ncsrGetNumberOfRows(csrFV) =", csrGetNumberOfRows(csrFV)@\\
\mbox{}\verb@print "\ncsrGetNumberOfRows(csrEV) =", csrGetNumberOfRows(csrEV)@\\
\mbox{}\verb@print "\n>>> csrGetNumberOfColumns"@\\
\mbox{}\verb@print "\ncsrGetNumberOfColumns(csrFV) =", csrGetNumberOfColumns(csrFV)@\\
\mbox{}\verb@print "\ncsrGetNumberOfColumns(csrEV) =", csrGetNumberOfColumns(csrEV)@\\
\mbox{}\verb@@{\NWsep}
\end{list}
\vspace{-1ex}
\footnotesize\addtolength{\baselineskip}{-1ex}
\begin{list}{}{\setlength{\itemsep}{-\parsep}\setlength{\itemindent}{-\leftmargin}}
\item \NWtxtMacroRefIn\ \NWlink{nuweb16b}{16b}.
\end{list}
\end{minipage}\\[4ex]
\end{flushleft}
%-------------------------------------------------------------------------------
%-------------------------------------------------------------------------------
\begin{flushleft} \small
\begin{minipage}{\linewidth} \label{scrap10}
\protect\makebox[0ex][r]{\NWtarget{nuweb5b}{\rule{0ex}{0ex}}\hspace{1em}}$\langle\,$Sparse to dense matrix transformation\nobreak\ {\footnotesize 5b}$\,\rangle\equiv$
\vspace{-1ex}
\begin{list}{}{} \item
\mbox{}\verb@def csrToMatrixRepresentation(CSRm):@\\
\mbox{}\verb@    nrows = csrGetNumberOfRows(CSRm)@\\
\mbox{}\verb@    ncolumns = csrGetNumberOfColumns(CSRm)@\\
\mbox{}\verb@    ScipyMat = zeros((nrows,ncolumns),int)@\\
\mbox{}\verb@    C = CSRm.tocoo()@\\
\mbox{}\verb@    for triple in zip(C.row,C.col,C.data):@\\
\mbox{}\verb@        ScipyMat[triple[0],triple[1]] = triple[2]@\\
\mbox{}\verb@    return ScipyMat@\\
\mbox{}\verb@@{\NWsep}
\end{list}
\vspace{-1ex}
\footnotesize\addtolength{\baselineskip}{-1ex}
\begin{list}{}{\setlength{\itemsep}{-\parsep}\setlength{\itemindent}{-\leftmargin}}
\item \NWtxtMacroRefIn\ \NWlink{nuweb16a}{16a}.
\end{list}
\end{minipage}\\[4ex]
\end{flushleft}
%-------------------------------------------------------------------------------
%-------------------------------------------------------------------------------
\begin{flushleft} \small
\begin{minipage}{\linewidth} \label{scrap11}
\protect\makebox[0ex][r]{\NWtarget{nuweb5c}{\rule{0ex}{0ex}}\hspace{1em}}$\langle\,$Test examples of Sparse to dense matrix transformation\nobreak\ {\footnotesize 5c}$\,\rangle\equiv$
\vspace{-1ex}
\begin{list}{}{} \item
\mbox{}\verb@print "\n>>> csrToMatrixRepresentation"@\\
\mbox{}\verb@print "\nFV =\n", csrToMatrixRepresentation(csrFV)@\\
\mbox{}\verb@print "\nEV =\n", csrToMatrixRepresentation(csrEV)@\\
\mbox{}\verb@@{\NWsep}
\end{list}
\vspace{-1ex}
\footnotesize\addtolength{\baselineskip}{-1ex}
\begin{list}{}{\setlength{\itemsep}{-\parsep}\setlength{\itemindent}{-\leftmargin}}
\item \NWtxtMacroRefIn\ \NWlink{nuweb16b}{16b}.
\end{list}
\end{minipage}\\[4ex]
\end{flushleft}
%-------------------------------------------------------------------------------
%-------------------------------------------------------------------------------
\begin{flushleft} \small
\begin{minipage}{\linewidth} \label{scrap12}
\protect\makebox[0ex][r]{\NWtarget{nuweb5d}{\rule{0ex}{0ex}}\hspace{1em}}$\langle\,$Matrix product and transposition\nobreak\ {\footnotesize 5d}$\,\rangle\equiv$
\vspace{-1ex}
\begin{list}{}{} \item
\mbox{}\verb@def matrixProduct(CSRm1,CSRm2):@\\
\mbox{}\verb@    CSRm = CSRm1 * CSRm2@\\
\mbox{}\verb@    return CSRm@\\
\mbox{}\verb@@\\
\mbox{}\verb@def csrTranspose(CSRm):@\\
\mbox{}\verb@    CSRm = CSRm.T@\\
\mbox{}\verb@    return CSRm@\\
\mbox{}\verb@@{\NWsep}
\end{list}
\vspace{-1ex}
\footnotesize\addtolength{\baselineskip}{-1ex}
\begin{list}{}{\setlength{\itemsep}{-\parsep}\setlength{\itemindent}{-\leftmargin}}
\item \NWtxtMacroRefIn\ \NWlink{nuweb16a}{16a}.
\end{list}
\end{minipage}\\[4ex]
\end{flushleft}
%-------------------------------------------------------------------------------
%-------------------------------------------------------------------------------
\begin{flushleft} \small
\begin{minipage}{\linewidth} \label{scrap13}
\protect\makebox[0ex][r]{\NWtarget{nuweb6a}{\rule{0ex}{0ex}}\hspace{1em}}$\langle\,$Matrix filtering to produce the boundary matrix\nobreak\ {\footnotesize 6a}$\,\rangle\equiv$
\vspace{-1ex}
\begin{list}{}{} \item
\mbox{}\verb@def csrBoundaryFilter(CSRm, facetLengths):@\\
\mbox{}\verb@    maxs = [max(CSRm[k].data) for k in range(CSRm.shape[0])]@\\
\mbox{}\verb@    inputShape = CSRm.shape@\\
\mbox{}\verb@    coo = CSRm.tocoo()@\\
\mbox{}\verb@    for k in range(len(coo.data)):@\\
\mbox{}\verb@        if coo.data[k]==maxs[coo.row[k]]: coo.data[k] = 1@\\
\mbox{}\verb@        else: coo.data[k] = 0@\\
\mbox{}\verb@    mtx = coo_matrix((coo.data, (coo.row, coo.col)), shape=inputShape)@\\
\mbox{}\verb@    out = mtx.tocsr()@\\
\mbox{}\verb@    return out@\\
\mbox{}\verb@@{\NWsep}
\end{list}
\vspace{-1ex}
\footnotesize\addtolength{\baselineskip}{-1ex}
\begin{list}{}{\setlength{\itemsep}{-\parsep}\setlength{\itemindent}{-\leftmargin}}
\item \NWtxtMacroRefIn\ \NWlink{nuweb16a}{16a}.
\end{list}
\end{minipage}\\[4ex]
\end{flushleft}
%-------------------------------------------------------------------------------
%-------------------------------------------------------------------------------
\begin{flushleft} \small
\begin{minipage}{\linewidth} \label{scrap14}
\protect\makebox[0ex][r]{\NWtarget{nuweb6b}{\rule{0ex}{0ex}}\hspace{1em}}$\langle\,$Test example of Matrix filtering to produce the boundary matrix\nobreak\ {\footnotesize 6b}$\,\rangle\equiv$
\vspace{-1ex}
\begin{list}{}{} \item
\mbox{}\verb@print "\n>>> csrBoundaryFilter"@\\
\mbox{}\verb@csrEF = matrixProduct(csrFV, csrTranspose(csrEV)).T@\\
\mbox{}\verb@facetLengths = [csrCell.getnnz() for csrCell in csrEV]@\\
\mbox{}\verb@CSRm = csrBoundaryFilter(csrEF, facetLengths).T@\\
\mbox{}\verb@print "\ncsrMaxFilter(csrFE) =\n", csrToMatrixRepresentation(CSRm)@\\
\mbox{}\verb@@{\NWsep}
\end{list}
\vspace{-1ex}
\footnotesize\addtolength{\baselineskip}{-1ex}
\begin{list}{}{\setlength{\itemsep}{-\parsep}\setlength{\itemindent}{-\leftmargin}}
\item \NWtxtMacroRefIn\ \NWlink{nuweb16b}{16b}.
\end{list}
\end{minipage}\\[4ex]
\end{flushleft}
%-------------------------------------------------------------------------------
%-------------------------------------------------------------------------------
\begin{flushleft} \small
\begin{minipage}{\linewidth} \label{scrap15}
\protect\makebox[0ex][r]{\NWtarget{nuweb6c}{\rule{0ex}{0ex}}\hspace{1em}}$\langle\,$Matrix filtering via a generic predicate\nobreak\ {\footnotesize 6c}$\,\rangle\equiv$
\vspace{-1ex}
\begin{list}{}{} \item
\mbox{}\verb@def csrPredFilter(CSRm, pred):@\\
\mbox{}\verb@   # can be done in parallel (by rows)@\\
\mbox{}\verb@   coo = CSRm.tocoo()@\\
\mbox{}\verb@   triples = [[row,col,val] for row,col,val @\\
\mbox{}\verb@            in zip(coo.row,coo.col,coo.data) if pred(val)]@\\
\mbox{}\verb@   i, j, data = TRANS(triples)@\\
\mbox{}\verb@   CSRm = scipy.sparse.coo_matrix((data,(i,j)),CSRm.shape).tocsr()@\\
\mbox{}\verb@   return CSRm@\\
\mbox{}\verb@@{\NWsep}
\end{list}
\vspace{-1ex}
\footnotesize\addtolength{\baselineskip}{-1ex}
\begin{list}{}{\setlength{\itemsep}{-\parsep}\setlength{\itemindent}{-\leftmargin}}
\item \NWtxtMacroRefIn\ \NWlink{nuweb16a}{16a}.
\end{list}
\end{minipage}\\[4ex]
\end{flushleft}
%-------------------------------------------------------------------------------
%-------------------------------------------------------------------------------
\begin{flushleft} \small
\begin{minipage}{\linewidth} \label{scrap16}
\protect\makebox[0ex][r]{\NWtarget{nuweb6d}{\rule{0ex}{0ex}}\hspace{1em}}$\langle\,$Test example of Matrix filtering via a generic predicate\nobreak\ {\footnotesize 6d}$\,\rangle\equiv$
\vspace{-1ex}
\begin{list}{}{} \item
\mbox{}\verb@print "\n>>> csrPredFilter"@\\
\mbox{}\verb@CSRm = csrPredFilter(matrixProduct(csrFV, csrTranspose(csrEV)).T, GE(2)).T@\\
\mbox{}\verb@print "\nccsrPredFilter(csrFE) =\n", csrToMatrixRepresentation(CSRm)@\\
\mbox{}\verb@@{\NWsep}
\end{list}
\vspace{-1ex}
\footnotesize\addtolength{\baselineskip}{-1ex}
\begin{list}{}{\setlength{\itemsep}{-\parsep}\setlength{\itemindent}{-\leftmargin}}
\item \NWtxtMacroRefIn\ \NWlink{nuweb16b}{16b}.
\end{list}
\end{minipage}\\[4ex]
\end{flushleft}
%-------------------------------------------------------------------------------

\section{Topological operations}

%-------------------------------------------------------------------------------
\begin{flushleft} \small
\begin{minipage}{\linewidth} \label{scrap17}
\protect\makebox[0ex][r]{\NWtarget{nuweb7a}{\rule{0ex}{0ex}}\hspace{1em}}$\langle\,$From cells and facets to boundary operator\nobreak\ {\footnotesize 7a}$\,\rangle\equiv$
\vspace{-1ex}
\begin{list}{}{} \item
\mbox{}\verb@def boundary(cells,facets):@\\
\mbox{}\verb@    csrCV = csrCreate(cells)@\\
\mbox{}\verb@    csrFV = csrCreate(facets)@\\
\mbox{}\verb@    csrFC = matrixProduct(csrFV, csrTranspose(csrCV))@\\
\mbox{}\verb@    facetLengths = [csrCell.getnnz() for csrCell in csrCV]@\\
\mbox{}\verb@    return csrBoundaryFilter(csrFC,facetLengths)@\\
\mbox{}\verb@@\\
\mbox{}\verb@def coboundary(cells,facets):@\\
\mbox{}\verb@    Boundary = boundary(cells,facets)@\\
\mbox{}\verb@    return csrTranspose(Boundary)@\\
\mbox{}\verb@@{\NWsep}
\end{list}
\vspace{-1ex}
\footnotesize\addtolength{\baselineskip}{-1ex}
\begin{list}{}{\setlength{\itemsep}{-\parsep}\setlength{\itemindent}{-\leftmargin}}
\item \NWtxtMacroRefIn\ \NWlink{nuweb16a}{16a}.
\end{list}
\end{minipage}\\[4ex]
\end{flushleft}
%-------------------------------------------------------------------------------
%-------------------------------------------------------------------------------
\begin{flushleft} \small
\begin{minipage}{\linewidth} \label{scrap18}
\protect\makebox[0ex][r]{\NWtarget{nuweb7b}{\rule{0ex}{0ex}}\hspace{1em}}$\langle\,$Test examples of From cells and facets to boundary operator\nobreak\ {\footnotesize 7b}$\,\rangle\equiv$
\vspace{-1ex}
\begin{list}{}{} \item
\mbox{}\verb@V = [[0.0, 0.0, 0.0], [1.0, 0.0, 0.0], [0.0, 1.0, 0.0], [1.0, 1.0, 0.0], @\\
\mbox{}\verb@[0.0, 0.0, 1.0], [1.0, 0.0, 1.0], [0.0, 1.0, 1.0], [1.0, 1.0, 1.0]]@\\
\mbox{}\verb@@\\
\mbox{}\verb@CV =[[0, 1, 2, 4], [1, 2, 4, 5], [2, 4, 5, 6], [1, 2, 3, 5], [2, 3, 5, 6], @\\
\mbox{}\verb@[3, 5, 6, 7]]@\\
\mbox{}\verb@@\\
\mbox{}\verb@FV =[[0, 1, 2], [0, 1, 4], [0, 2, 4], [1, 2, 3], [1, 2, 4], [1, 2, 5], @\\
\mbox{}\verb@[1, 3, 5], [1, 4, 5], [2, 3, 5], [2, 3, 6], [2, 4, 5], [2, 4, 6], [2, 5, 6], @\\
\mbox{}\verb@[3, 5, 6], [3, 5, 7], [3, 6, 7], [4, 5, 6], [5, 6, 7]]@\\
\mbox{}\verb@@\\
\mbox{}\verb@EV =[[0, 1], [0, 2], [0, 4], [1, 2], [1, 3], [1, 4], [1, 5], [2, 3], [2, 4], @\\
\mbox{}\verb@[2, 5], [2, 6], [3, 5], [3, 6], [3, 7], [4, 5], [4, 6], [5, 6], [5, 7], @\\
\mbox{}\verb@[6, 7]]@\\
\mbox{}\verb@@\\
\mbox{}\verb@print "\ncoboundary_2 =\n", csrToMatrixRepresentation(coboundary(CV,FV))@\\
\mbox{}\verb@print "\ncoboundary_1 =\n", csrToMatrixRepresentation(coboundary(FV,EV))@\\
\mbox{}\verb@print "\ncoboundary_0 =\n", csrToMatrixRepresentation(coboundary(EV,AA(LIST)(range(len(V)))))@\\
\mbox{}\verb@@{\NWsep}
\end{list}
\vspace{-1ex}
\footnotesize\addtolength{\baselineskip}{-1ex}
\begin{list}{}{\setlength{\itemsep}{-\parsep}\setlength{\itemindent}{-\leftmargin}}
\item \NWtxtMacroRefIn\ \NWlink{nuweb16b}{16b}.
\end{list}
\end{minipage}\\[4ex]
\end{flushleft}
%-------------------------------------------------------------------------------
%-------------------------------------------------------------------------------
\begin{flushleft} \small
\begin{minipage}{\linewidth} \label{scrap19}
\protect\makebox[0ex][r]{\NWtarget{nuweb8a}{\rule{0ex}{0ex}}\hspace{1em}}$\langle\,$From cells and facets to boundary cells\nobreak\ {\footnotesize 8a}$\,\rangle\equiv$
\vspace{-1ex}
\begin{list}{}{} \item
\mbox{}\verb@def zeroChain(cells):@\\
\mbox{}\verb@   pass@\\
\mbox{}\verb@@\\
\mbox{}\verb@def totalChain(cells):@\\
\mbox{}\verb@   return csrCreate([[0] for cell in cells])@\\
\mbox{}\verb@@\\
\mbox{}\verb@def boundaryCells(cells,facets):@\\
\mbox{}\verb@   csrBoundaryMat = boundary(cells,facets)@\\
\mbox{}\verb@   csrChain = totalChain(cells)@\\
\mbox{}\verb@   csrBoundaryChain = matrixProduct(csrBoundaryMat, csrChain)@\\
\mbox{}\verb@   for k,value in enumerate(csrBoundaryChain.data):@\\
\mbox{}\verb@      if value % 2 == 0: csrBoundaryChain.data[k] = 0@\\
\mbox{}\verb@   boundaryCells = [k for k,val in enumerate(csrBoundaryChain.data.tolist()) if val == 1]@\\
\mbox{}\verb@   return boundaryCells@\\
\mbox{}\verb@@{\NWsep}
\end{list}
\vspace{-1ex}
\footnotesize\addtolength{\baselineskip}{-1ex}
\begin{list}{}{\setlength{\itemsep}{-\parsep}\setlength{\itemindent}{-\leftmargin}}
\item \NWtxtMacroRefIn\ \NWlink{nuweb16a}{16a}.
\end{list}
\end{minipage}\\[4ex]
\end{flushleft}
%-------------------------------------------------------------------------------
%-------------------------------------------------------------------------------
\begin{flushleft} \small
\begin{minipage}{\linewidth} \label{scrap20}
\protect\makebox[0ex][r]{\NWtarget{nuweb8b}{\rule{0ex}{0ex}}\hspace{1em}}$\langle\,$Test examples of From cells and facets to boundary cells\nobreak\ {\footnotesize 8b}$\,\rangle\equiv$
\vspace{-1ex}
\begin{list}{}{} \item
\mbox{}\verb@boundaryCells_2 = boundaryCells(CV,FV)@\\
\mbox{}\verb@boundaryCells_1 = boundaryCells([FV[k] for k in boundaryCells_2],EV)@\\
\mbox{}\verb@@\\
\mbox{}\verb@print "\nboundaryCells_2 =\n", boundaryCells_2@\\
\mbox{}\verb@print "\nboundaryCells_1 =\n", boundaryCells_1@\\
\mbox{}\verb@@\\
\mbox{}\verb@boundary = (V,[FV[k] for k in boundaryCells_2])@\\
\mbox{}\verb@VIEW(EXPLODE(1.5,1.5,1.5)(MKPOLS(boundary)))@\\
\mbox{}\verb@@{\NWsep}
\end{list}
\vspace{-1ex}
\footnotesize\addtolength{\baselineskip}{-1ex}
\begin{list}{}{\setlength{\itemsep}{-\parsep}\setlength{\itemindent}{-\leftmargin}}
\item \NWtxtMacroRefIn\ \NWlink{nuweb16b}{16b}.
\end{list}
\end{minipage}\\[4ex]
\end{flushleft}
%-------------------------------------------------------------------------------
%-------------------------------------------------------------------------------
\begin{flushleft} \small
\begin{minipage}{\linewidth} \label{scrap21}
\protect\makebox[0ex][r]{\NWtarget{nuweb9a}{\rule{0ex}{0ex}}\hspace{1em}}$\langle\,$Signed boundary matrix for simplicial models\nobreak\ {\footnotesize 9a}$\,\rangle\equiv$
\vspace{-1ex}
\begin{list}{}{} \item
\mbox{}\verb@def signedBoundary (V,CV,FV):@\\
\mbox{}\verb@   # compute the set of pairs of indices to [boundary face,incident coface]@\\
\mbox{}\verb@   coo = boundary(CV,FV).tocoo()@\\
\mbox{}\verb@   pairs = [[coo.row[k],coo.col[k]] for k,val in enumerate(coo.data) if val != 0]@\\
\mbox{}\verb@   @\\
\mbox{}\verb@   # compute the [face, coface] pair as vertex lists@\\
\mbox{}\verb@   vertLists = [[FV[pair[0]], CV[pair[1]]]for pair in pairs]@\\
\mbox{}\verb@   @\\
\mbox{}\verb@   # compute two n-cells to compare for sign@\\
\mbox{}\verb@   cellPairs = [ [list(set(coface).difference(face))+face,coface] @\\
\mbox{}\verb@               for face,coface in vertLists]@\\
\mbox{}\verb@   @\\
\mbox{}\verb@   # compute the local indices of missing boundary cofaces@\\
\mbox{}\verb@   missingVertIndices = [ coface.index(list(set(coface).difference(face))[0]) @\\
\mbox{}\verb@                     for face,coface in vertLists]@\\
\mbox{}\verb@   @\\
\mbox{}\verb@   # compute the point matrices to compare for sign@\\
\mbox{}\verb@   pointArrays = [ [[V[k]+[1.0] for k in facetCell], [V[k]+[1.0] for k in cofaceCell]] @\\
\mbox{}\verb@               for facetCell,cofaceCell in cellPairs]@\\
\mbox{}\verb@   @\\
\mbox{}\verb@   # signed incidence coefficients@\\
\mbox{}\verb@   cofaceMats = TRANS(pointArrays)[1]@\\
\mbox{}\verb@   cofaceSigns = AA(SIGN)(AA(np.linalg.det)(cofaceMats))@\\
\mbox{}\verb@   faceSigns = AA(C(POWER)(-1))(missingVertIndices)@\\
\mbox{}\verb@   signPairProd = AA(PROD)(TRANS([cofaceSigns,faceSigns]))@\\
\mbox{}\verb@   @\\
\mbox{}\verb@   # signed boundary matrix@\\
\mbox{}\verb@   csrSignedBoundaryMat = csr_matrix( (signPairProd,TRANS(pairs)) )@\\
\mbox{}\verb@   return csrSignedBoundaryMat@\\
\mbox{}\verb@@{\NWsep}
\end{list}
\vspace{-1ex}
\footnotesize\addtolength{\baselineskip}{-1ex}
\begin{list}{}{\setlength{\itemsep}{-\parsep}\setlength{\itemindent}{-\leftmargin}}
\item \NWtxtMacroRefIn\ \NWlink{nuweb16a}{16a}.
\end{list}
\end{minipage}\\[4ex]
\end{flushleft}
%-------------------------------------------------------------------------------
%-------------------------------------------------------------------------------
\begin{flushleft} \small
\begin{minipage}{\linewidth} \label{scrap22}
\protect\makebox[0ex][r]{\NWtarget{nuweb9b}{\rule{0ex}{0ex}}\hspace{1em}}$\langle\,$Oriented boundary cells for simplicial models\nobreak\ {\footnotesize 9b}$\,\rangle\equiv$
\vspace{-1ex}
\begin{list}{}{} \item
\mbox{}\verb@def signedBoundaryCells(verts,cells,facets):@\\
\mbox{}\verb@   csrBoundaryMat = signedBoundary(verts,cells,facets)@\\
\mbox{}\verb@   csrTotalChain = totalChain(cells)@\\
\mbox{}\verb@   csrBoundaryChain = matrixProduct(csrBoundaryMat, csrTotalChain)@\\
\mbox{}\verb@   coo = csrBoundaryChain.tocoo()@\\
\mbox{}\verb@   boundaryCells = list(coo.row * coo.data)@\\
\mbox{}\verb@   return AA(int)(boundaryCells)@\\
\mbox{}\verb@@{\NWsep}
\end{list}
\vspace{-1ex}
\footnotesize\addtolength{\baselineskip}{-1ex}
\begin{list}{}{\setlength{\itemsep}{-\parsep}\setlength{\itemindent}{-\leftmargin}}
\item \NWtxtMacroDefBy\ \NWlink{nuweb9b}{9b}\NWlink{nuweb11}{, 11}.
\item \NWtxtMacroRefIn\ \NWlink{nuweb16a}{16a}.
\end{list}
\end{minipage}\\[4ex]
\end{flushleft}
%-------------------------------------------------------------------------------
\paragraph{Orienting polytopal cells}
\begin{description}
	\item[input]:  "cell" indices of a convex and solid polytopes and "V" vertices;
	\item[output]:  biggest "simplex" indices spanning the polytope.
	\item[\tt m]: number of cell vertices
	\item[\tt d]: dimension (number of coordinates) of cell vertices
	\item[\tt d+1]: number of simplex vertices
	\item[\tt vcell]: cell vertices
	\item[\tt vsimplex]: simplex vertices
	\item[\tt Id]: identity matrix
	\item[\tt basis]: orthonormal spanning set of vectors $e_k$
	\item[\tt vector]: position vector of a simplex vertex in translated coordinates
	\item[\tt unUsedIndices]: cell indices not moved to simplex
\end{description}

%-------------------------------------------------------------------------------
\begin{flushleft} \small
\begin{minipage}{\linewidth} \label{scrap23}
\protect\makebox[0ex][r]{\NWtarget{nuweb11}{\rule{0ex}{0ex}}\hspace{1em}}$\langle\,$Oriented boundary cells for simplicial models\nobreak\ {\footnotesize 11}$\,\rangle\equiv$
\vspace{-1ex}
\begin{list}{}{} \item
\mbox{}\verb@def pivotSimplices(V,CV,d=3):@\\
\mbox{}\verb@   simplices = []@\\
\mbox{}\verb@   for cell in CV:@\\
\mbox{}\verb@      vcell = np.array([V[v] for v in cell])@\\
\mbox{}\verb@      m, simplex = len(cell), []@\\
\mbox{}\verb@      # translate the cell: for each k, vcell[k] -= vcell[0], and simplex[0] := cell[0]@\\
\mbox{}\verb@      for k in range(m-1,-1,-1): vcell[k] -= vcell[0]@\\
\mbox{}\verb@      # simplex = [0], basis = [], tensor = Id(d+1)@\\
\mbox{}\verb@      simplex += [cell[0]]@\\
\mbox{}\verb@      basis = []@\\
\mbox{}\verb@      tensor = np.array(IDNT(d))@\\
\mbox{}\verb@      # look for most far cell vertex@\\
\mbox{}\verb@      dists = [SUM([SQR(x) for x in v])**0.5 for v in vcell]@\\
\mbox{}\verb@      maxDistIndex = max(enumerate(dists),key=lambda x: x[1])[0]@\\
\mbox{}\verb@      vector = np.array([vcell[maxDistIndex]])@\\
\mbox{}\verb@      # normalize vector@\\
\mbox{}\verb@      den=(vector**2).sum(axis=-1) **0.5@\\
\mbox{}\verb@      basis = [vector/den]@\\
\mbox{}\verb@      simplex += [cell[maxDistIndex]]@\\
\mbox{}\verb@      unUsedIndices = [h for h in cell if h not in simplex]@\\
\mbox{}\verb@      @\\
\mbox{}\verb@      # for k in {2,d+1}:@\\
\mbox{}\verb@      for k in range(2,d+1):@\\
\mbox{}\verb@         # update the orthonormal tensor@\\
\mbox{}\verb@         e = basis[-1]@\\
\mbox{}\verb@         tensor = tensor - np.dot(e.T, e)@\\
\mbox{}\verb@         # compute the index h of a best vector@\\
\mbox{}\verb@         # look for most far cell vertex@\\
\mbox{}\verb@         dists = [SUM([SQR(x) for x in np.dot(tensor,v)])**0.5@\\
\mbox{}\verb@         if h in unUsedIndices else 0.0@\\
\mbox{}\verb@         for (h,v) in zip(cell,vcell)]@\\
\mbox{}\verb@         # insert the best vector index h in output simplex@\\
\mbox{}\verb@         maxDistIndex = max(enumerate(dists),key=lambda x: x[1])[0]@\\
\mbox{}\verb@         vector = np.array([vcell[maxDistIndex]])@\\
\mbox{}\verb@         # normalize vector@\\
\mbox{}\verb@         den=(vector**2).sum(axis=-1) **0.5@\\
\mbox{}\verb@         basis += [vector/den]@\\
\mbox{}\verb@         simplex += [cell[maxDistIndex]]@\\
\mbox{}\verb@         unUsedIndices = [h for h in cell if h not in simplex]@\\
\mbox{}\verb@      simplices += [simplex]@\\
\mbox{}\verb@   return simplices@\\
\mbox{}\verb@@\\
\mbox{}\verb@def simplexOrientations(V,simplices):@\\
\mbox{}\verb@   vcells = [[V[v]+[1.0] for v in simplex] for simplex in simplices]@\\
\mbox{}\verb@   return [SIGN(np.linalg.det(vcell)) for vcell in vcells]@\\
\mbox{}\verb@@{\NWsep}
\end{list}
\vspace{-1ex}
\footnotesize\addtolength{\baselineskip}{-1ex}
\begin{list}{}{\setlength{\itemsep}{-\parsep}\setlength{\itemindent}{-\leftmargin}}
\item \NWtxtMacroDefBy\ \NWlink{nuweb9b}{9b}\NWlink{nuweb11}{, 11}.
\item \NWtxtMacroRefIn\ \NWlink{nuweb16a}{16a}.
\end{list}
\end{minipage}\\[4ex]
\end{flushleft}
%-------------------------------------------------------------------------------
%-------------------------------------------------------------------------------
\begin{flushleft} \small
\begin{minipage}{\linewidth} \label{scrap24}
\protect\makebox[0ex][r]{\NWtarget{nuweb12a}{\rule{0ex}{0ex}}\hspace{1em}}$\langle\,$Computation of cell adjacencies\nobreak\ {\footnotesize 12a}$\,\rangle\equiv$
\vspace{-1ex}
\begin{list}{}{} \item
\mbox{}\verb@def larCellAdjacencies(CSRm):@\\
\mbox{}\verb@    CSRm = matrixProduct(CSRm,csrTranspose(CSRm))@\\
\mbox{}\verb@    return CSRm@\\
\mbox{}\verb@@{\NWsep}
\end{list}
\vspace{-1ex}
\footnotesize\addtolength{\baselineskip}{-1ex}
\begin{list}{}{\setlength{\itemsep}{-\parsep}\setlength{\itemindent}{-\leftmargin}}
\item \NWtxtMacroRefIn\ \NWlink{nuweb16a}{16a}.
\end{list}
\end{minipage}\\[4ex]
\end{flushleft}
%-------------------------------------------------------------------------------
%-------------------------------------------------------------------------------
\begin{flushleft} \small
\begin{minipage}{\linewidth} \label{scrap25}
\protect\makebox[0ex][r]{\NWtarget{nuweb12b}{\rule{0ex}{0ex}}\hspace{1em}}$\langle\,$Test examples of Computation of cell adjacencies\nobreak\ {\footnotesize 12b}$\,\rangle\equiv$
\vspace{-1ex}
\begin{list}{}{} \item
\mbox{}\verb@print "\n>>> larCellAdjacencies"@\\
\mbox{}\verb@adj_2_cells = larCellAdjacencies(csrFV)@\\
\mbox{}\verb@print "\nadj_2_cells =\n", csrToMatrixRepresentation(adj_2_cells)@\\
\mbox{}\verb@adj_1_cells = larCellAdjacencies(csrEV)@\\
\mbox{}\verb@print "\nadj_1_cells =\n", csrToMatrixRepresentation(adj_1_cells)@\\
\mbox{}\verb@@{\NWsep}
\end{list}
\vspace{-1ex}
\footnotesize\addtolength{\baselineskip}{-1ex}
\begin{list}{}{\setlength{\itemsep}{-\parsep}\setlength{\itemindent}{-\leftmargin}}
\item \NWtxtMacroRefIn\ \NWlink{nuweb16b}{16b}.
\end{list}
\end{minipage}\\[4ex]
\end{flushleft}
%-------------------------------------------------------------------------------
%-------------------------------------------------------------------------------
\begin{flushleft} \small
\begin{minipage}{\linewidth} \label{scrap26}
\protect\makebox[0ex][r]{\NWtarget{nuweb13}{\rule{0ex}{0ex}}\hspace{1em}}$\langle\,$Extraction of facets of a cell complex\nobreak\ {\footnotesize 13}$\,\rangle\equiv$
\vspace{-1ex}
\begin{list}{}{} \item
\mbox{}\verb@def setup(model,dim):@\\
\mbox{}\verb@    V, cells = model@\\
\mbox{}\verb@    csr = csrCreate(cells)@\\
\mbox{}\verb@    csrAdjSquareMat = larCellAdjacencies(csr)@\\
\mbox{}\verb@    csrAdjSquareMat = csrPredFilter(csrAdjSquareMat, GE(dim)) # ? HOWTODO ?@\\
\mbox{}\verb@    return V,cells,csr,csrAdjSquareMat@\\
\mbox{}\verb@@\\
\mbox{}\verb@def larFacets(model,dim=3):@\\
\mbox{}\verb@    """@\\
\mbox{}\verb@        Estraction of (d-1)-cellFacets from "model" := (V,d-cells)@\\
\mbox{}\verb@        Return (V, (d-1)-cellFacets)@\\
\mbox{}\verb@      """@\\
\mbox{}\verb@    V,cells,csr,csrAdjSquareMat = setup(model,dim)@\\
\mbox{}\verb@    cellFacets = []@\\
\mbox{}\verb@    # for each input cell i@\\
\mbox{}\verb@    for i in range(len(cells)):@\\
\mbox{}\verb@        adjCells = csrAdjSquareMat[i].tocoo()@\\
\mbox{}\verb@        cell1 = csr[i].tocoo().col@\\
\mbox{}\verb@        pairs = zip(adjCells.col,adjCells.data)@\\
\mbox{}\verb@        for j,v in pairs:@\\
\mbox{}\verb@            if (i<j):@\\
\mbox{}\verb@                cell2 = csr[j].tocoo().col@\\
\mbox{}\verb@                cell = list(set(cell1).intersection(cell2))@\\
\mbox{}\verb@                cellFacets.append(sorted(cell))@\\
\mbox{}\verb@    # sort and remove duplicates@\\
\mbox{}\verb@    cellFacets = sorted(AA(list)(set(AA(tuple)(cellFacets))))@\\
\mbox{}\verb@    return V,cellFacets@\\
\mbox{}\verb@@{\NWsep}
\end{list}
\vspace{-1ex}
\footnotesize\addtolength{\baselineskip}{-1ex}
\begin{list}{}{\setlength{\itemsep}{-\parsep}\setlength{\itemindent}{-\leftmargin}}
\item \NWtxtMacroRefIn\ \NWlink{nuweb16a}{16a}.
\end{list}
\end{minipage}\\[4ex]
\end{flushleft}
%-------------------------------------------------------------------------------
%-------------------------------------------------------------------------------
\begin{flushleft} \small
\begin{minipage}{\linewidth} \label{scrap27}
\protect\makebox[0ex][r]{\NWtarget{nuweb14}{\rule{0ex}{0ex}}\hspace{1em}}$\langle\,$Test examples of Extraction of facets of a cell complex\nobreak\ {\footnotesize 14}$\,\rangle\equiv$
\vspace{-1ex}
\begin{list}{}{} \item
\mbox{}\verb@V = [[0.,0.],[3.,0.],[0.,3.],[3.,3.],[1.,2.],[2.,2.],[1.,1.],[2.,1.]]@\\
\mbox{}\verb@FV = [[0,1,6,7],[0,2,4,6],[4,5,6,7],[1,3,5,7],[2,3,4,5],[0,1,2,3]]@\\
\mbox{}\verb@@\\
\mbox{}\verb@_,EV = larFacets((V,FV),dim=2)@\\
\mbox{}\verb@print "\nEV =",EV@\\
\mbox{}\verb@VIEW(EXPLODE(1.5,1.5,1.5)(MKPOLS((V,EV))))@\\
\mbox{}\verb@@\\
\mbox{}\verb@FV = [[0,1,3],[1,2,4],[2,4,5],[3,4,6],[4,6,7],[5,7,8], # full@\\
\mbox{}\verb@   [1,3,4],[4,5,7], # empty@\\
\mbox{}\verb@   [0,1,2],[6,7,8],[0,3,6],[2,5,8]] # exterior@\\
\mbox{}\verb@      @\\
\mbox{}\verb@_,EV = larFacets((V,FV),dim=2)@\\
\mbox{}\verb@print "\nEV =",EV@\\
\mbox{}\verb@@{\NWsep}
\end{list}
\vspace{-1ex}
\footnotesize\addtolength{\baselineskip}{-1ex}
\begin{list}{}{\setlength{\itemsep}{-\parsep}\setlength{\itemindent}{-\leftmargin}}
\item \NWtxtMacroRefIn\ \NWlink{nuweb16b}{16b}.
\end{list}
\end{minipage}\\[4ex]
\end{flushleft}
%-------------------------------------------------------------------------------

\section{Exporting the library}

\subsection{MIT licence}
%-------------------------------------------------------------------------------
\begin{flushleft} \small
\begin{minipage}{\linewidth} \label{scrap28}
\protect\makebox[0ex][r]{\NWtarget{nuweb15a}{\rule{0ex}{0ex}}\hspace{1em}}$\langle\,$The MIT Licence\nobreak\ {\footnotesize 15a}$\,\rangle\equiv$
\vspace{-1ex}
\begin{list}{}{} \item
\mbox{}\verb@@\\
\mbox{}\verb@"""@\\
\mbox{}\verb@The MIT License@\\
\mbox{}\verb@===============@\\
\mbox{}\verb@    @\\
\mbox{}\verb@Permission is hereby granted, free of charge, to any person obtaining@\\
\mbox{}\verb@a copy of this software and associated documentation files (the@\\
\mbox{}\verb@'Software'), to deal in the Software without restriction, including@\\
\mbox{}\verb@without limitation the rights to use, copy, modify, merge, publish,@\\
\mbox{}\verb@distribute, sublicense, and/or sell copies of the Software, and to@\\
\mbox{}\verb@permit persons to whom the Software is furnished to do so, subject to@\\
\mbox{}\verb@the following conditions:@\\
\mbox{}\verb@@\\
\mbox{}\verb@The above copyright notice and this permission notice shall be@\\
\mbox{}\verb@included in all copies or substantial portions of the Software.@\\
\mbox{}\verb@@\\
\mbox{}\verb@THE SOFTWARE IS PROVIDED 'AS IS', WITHOUT WARRANTY OF ANY KIND,@\\
\mbox{}\verb@EXPRESS OR IMPLIED, INCLUDING BUT NOT LIMITED TO THE WARRANTIES OF@\\
\mbox{}\verb@MERCHANTABILITY, FITNESS FOR A PARTICULAR PURPOSE AND NONINFRINGEMENT.@\\
\mbox{}\verb@IN NO EVENT SHALL THE AUTHORS OR COPYRIGHT HOLDERS BE LIABLE FOR ANY@\\
\mbox{}\verb@CLAIM, DAMAGES OR OTHER LIABILITY, WHETHER IN AN ACTION OF CONTRACT,@\\
\mbox{}\verb@TORT OR OTHERWISE, ARISING FROM, OUT OF OR IN CONNECTION WITH THE@\\
\mbox{}\verb@SOFTWARE OR THE USE OR OTHER DEALINGS IN THE SOFTWARE.@\\
\mbox{}\verb@"""@\\
\mbox{}\verb@@{\NWsep}
\end{list}
\vspace{-1ex}
\footnotesize\addtolength{\baselineskip}{-1ex}
\begin{list}{}{\setlength{\itemsep}{-\parsep}\setlength{\itemindent}{-\leftmargin}}
\item \NWtxtMacroRefIn\ \NWlink{nuweb16a}{16a}.
\end{list}
\end{minipage}\\[4ex]
\end{flushleft}
%-------------------------------------------------------------------------------
\subsection{Importing of modules or packages}
%-------------------------------------------------------------------------------
\begin{flushleft} \small
\begin{minipage}{\linewidth} \label{scrap29}
\protect\makebox[0ex][r]{\NWtarget{nuweb15b}{\rule{0ex}{0ex}}\hspace{1em}}$\langle\,$Importing of modules or packages\nobreak\ {\footnotesize 15b}$\,\rangle\equiv$
\vspace{-1ex}
\begin{list}{}{} \item
\mbox{}\verb@from pyplasm import *@\\
\mbox{}\verb@import collections@\\
\mbox{}\verb@import scipy@\\
\mbox{}\verb@import numpy as np@\\
\mbox{}\verb@from scipy import zeros,arange,mat,amin,amax@\\
\mbox{}\verb@from scipy.sparse import vstack,hstack,csr_matrix,coo_matrix,lil_matrix,triu@\\
\mbox{}\verb@@\\
\mbox{}\verb@from lar2psm import *@\\
\mbox{}\verb@@{\NWsep}
\end{list}
\vspace{-1ex}
\footnotesize\addtolength{\baselineskip}{-1ex}
\begin{list}{}{\setlength{\itemsep}{-\parsep}\setlength{\itemindent}{-\leftmargin}}
\item \NWtxtMacroRefIn\ \NWlink{nuweb16a}{16a}.
\end{list}
\end{minipage}\\[4ex]
\end{flushleft}
%-------------------------------------------------------------------------------

\subsection{Writing the library file}

%-------------------------------------------------------------------------------
\begin{flushleft} \small
\begin{minipage}{\linewidth} \label{scrap30}
\protect\makebox[0ex][r]{\NWtarget{nuweb16a}{\rule{0ex}{0ex}}\hspace{1em}}\verb@"lib/py/larcc.py"@\nobreak\ {\footnotesize 16a }$\equiv$
\vspace{-1ex}
\begin{list}{}{} \item
\mbox{}\verb@# -*- coding: utf-8 -*-@\\
\mbox{}\verb@""" Basic LARCC library """@\\
\mbox{}\verb@@\hbox{$\langle\,$The MIT Licence\nobreak\ {\footnotesize \NWlink{nuweb15a}{15a}}$\,\rangle$}\verb@@\\
\mbox{}\verb@@\hbox{$\langle\,$Importing of modules or packages\nobreak\ {\footnotesize \NWlink{nuweb15b}{15b}}$\,\rangle$}\verb@@\\
\mbox{}\verb@@\hbox{$\langle\,$From list of triples to scipy.sparse\nobreak\ {\footnotesize \NWlink{nuweb3a}{3a}}$\,\rangle$}\verb@@\\
\mbox{}\verb@@\hbox{$\langle\,$Brc to Coo transformation\nobreak\ {\footnotesize \NWlink{nuweb3b}{3b}}$\,\rangle$}\verb@@\\
\mbox{}\verb@@\hbox{$\langle\,$Coo to Csr transformation\nobreak\ {\footnotesize \NWlink{nuweb3d}{3d}}$\,\rangle$}\verb@@\\
\mbox{}\verb@@\hbox{$\langle\,$Brc to Csr transformation\nobreak\ {\footnotesize \NWlink{nuweb4b}{4b}}$\,\rangle$}\verb@@\\
\mbox{}\verb@@\hbox{$\langle\,$Query Matrix shape\nobreak\ {\footnotesize \NWlink{nuweb4d}{4d}}$\,\rangle$}\verb@@\\
\mbox{}\verb@@\hbox{$\langle\,$Sparse to dense matrix transformation\nobreak\ {\footnotesize \NWlink{nuweb5b}{5b}}$\,\rangle$}\verb@@\\
\mbox{}\verb@@\hbox{$\langle\,$Matrix product and transposition\nobreak\ {\footnotesize \NWlink{nuweb5d}{5d}}$\,\rangle$}\verb@@\\
\mbox{}\verb@@\hbox{$\langle\,$Matrix filtering to produce the boundary matrix\nobreak\ {\footnotesize \NWlink{nuweb6a}{6a}}$\,\rangle$}\verb@@\\
\mbox{}\verb@@\hbox{$\langle\,$Matrix filtering via a generic predicate\nobreak\ {\footnotesize \NWlink{nuweb6c}{6c}}$\,\rangle$}\verb@@\\
\mbox{}\verb@@\hbox{$\langle\,$From cells and facets to boundary operator\nobreak\ {\footnotesize \NWlink{nuweb7a}{7a}}$\,\rangle$}\verb@@\\
\mbox{}\verb@@\hbox{$\langle\,$From cells and facets to boundary cells\nobreak\ {\footnotesize \NWlink{nuweb8a}{8a}}$\,\rangle$}\verb@@\\
\mbox{}\verb@@\hbox{$\langle\,$Signed boundary matrix for simplicial models\nobreak\ {\footnotesize \NWlink{nuweb9a}{9a}}$\,\rangle$}\verb@@\\
\mbox{}\verb@@\hbox{$\langle\,$Oriented boundary cells for simplicial models\nobreak\ {\footnotesize \NWlink{nuweb9b}{9b}, \ldots\ }$\,\rangle$}\verb@@\\
\mbox{}\verb@@\hbox{$\langle\,$Computation of cell adjacencies\nobreak\ {\footnotesize \NWlink{nuweb12a}{12a}}$\,\rangle$}\verb@@\\
\mbox{}\verb@@\hbox{$\langle\,$Extraction of facets of a cell complex\nobreak\ {\footnotesize \NWlink{nuweb13}{13}}$\,\rangle$}\verb@@\\
\mbox{}\verb@@\\
\mbox{}\verb@if __name__ == "__main__": @\\
\mbox{}\verb@   @\hbox{$\langle\,$Test examples\nobreak\ {\footnotesize \NWlink{nuweb16b}{16b}}$\,\rangle$}\verb@@\\
\mbox{}\verb@@{\NWsep}
\end{list}
\vspace{-2ex}
\end{minipage}\\[4ex]
\end{flushleft}
%-------------------------------------------------------------------------------

\section{Unit tests}


%-------------------------------------------------------------------------------
\begin{flushleft} \small
\begin{minipage}{\linewidth} \label{scrap31}
\protect\makebox[0ex][r]{\NWtarget{nuweb16b}{\rule{0ex}{0ex}}\hspace{1em}}$\langle\,$Test examples\nobreak\ {\footnotesize 16b}$\,\rangle\equiv$
\vspace{-1ex}
\begin{list}{}{} \item
\mbox{}\verb@@\\
\mbox{}\verb@@\hbox{$\langle\,$Test example of Brc to Coo transformation\nobreak\ {\footnotesize \NWlink{nuweb3c}{3c}}$\,\rangle$}\verb@@\\
\mbox{}\verb@@\hbox{$\langle\,$Test example of Coo to Csr transformation\nobreak\ {\footnotesize \NWlink{nuweb4a}{4a}}$\,\rangle$}\verb@@\\
\mbox{}\verb@@\hbox{$\langle\,$Test example of Brc to Csr transformation\nobreak\ {\footnotesize \NWlink{nuweb4c}{4c}}$\,\rangle$}\verb@@\\
\mbox{}\verb@@\hbox{$\langle\,$Test examples of Query Matrix shape\nobreak\ {\footnotesize \NWlink{nuweb5a}{5a}}$\,\rangle$}\verb@@\\
\mbox{}\verb@@\hbox{$\langle\,$Test examples of Sparse to dense matrix transformation\nobreak\ {\footnotesize \NWlink{nuweb5c}{5c}}$\,\rangle$}\verb@@\\
\mbox{}\verb@@\hbox{$\langle\,$Test example of Matrix filtering to produce the boundary matrix\nobreak\ {\footnotesize \NWlink{nuweb6b}{6b}}$\,\rangle$}\verb@@\\
\mbox{}\verb@@\hbox{$\langle\,$Test example of Matrix filtering via a generic predicate\nobreak\ {\footnotesize \NWlink{nuweb6d}{6d}}$\,\rangle$}\verb@@\\
\mbox{}\verb@@\hbox{$\langle\,$Test examples of From cells and facets to boundary operator\nobreak\ {\footnotesize \NWlink{nuweb7b}{7b}}$\,\rangle$}\verb@@\\
\mbox{}\verb@@\hbox{$\langle\,$Test examples of From cells and facets to boundary cells\nobreak\ {\footnotesize \NWlink{nuweb8b}{8b}}$\,\rangle$}\verb@@\\
\mbox{}\verb@@\hbox{$\langle\,$Test examples of Computation of cell adjacencies\nobreak\ {\footnotesize \NWlink{nuweb12b}{12b}}$\,\rangle$}\verb@@\\
\mbox{}\verb@@\hbox{$\langle\,$Test examples of Extraction of facets of a cell complex\nobreak\ {\footnotesize \NWlink{nuweb14}{14}}$\,\rangle$}\verb@@\\
\mbox{}\verb@@{\NWsep}
\end{list}
\vspace{-1ex}
\footnotesize\addtolength{\baselineskip}{-1ex}
\begin{list}{}{\setlength{\itemsep}{-\parsep}\setlength{\itemindent}{-\leftmargin}}
\item \NWtxtMacroRefIn\ \NWlink{nuweb16a}{16a}.
\end{list}
\end{minipage}\\[4ex]
\end{flushleft}
%-------------------------------------------------------------------------------


\appendix

\section{Appendix: Tutorials}


\subsection{Model generation, skeleton and boundary extraction}

%-------------------------------------------------------------------------------
\begin{flushleft} \small
\begin{minipage}{\linewidth} \label{scrap32}
\protect\makebox[0ex][r]{\NWtarget{nuweb17a}{\rule{0ex}{0ex}}\hspace{1em}}\verb@"test/py/larcc/ex1.py"@\nobreak\ {\footnotesize 17a }$\equiv$
\vspace{-1ex}
\begin{list}{}{} \item
\mbox{}\verb@@\\
\mbox{}\verb@from larcc import *@\\
\mbox{}\verb@from largrid import *@\\
\mbox{}\verb@@\hbox{$\langle\,$input of 2D topology and geometry data\nobreak\ {\footnotesize \NWlink{nuweb17b}{17b}}$\,\rangle$}\verb@@\\
\mbox{}\verb@@\hbox{$\langle\,$characteristic matrices\nobreak\ {\footnotesize \NWlink{nuweb17c}{17c}}$\,\rangle$}\verb@@\\
\mbox{}\verb@@\hbox{$\langle\,$incidence matrix\nobreak\ {\footnotesize \NWlink{nuweb17d}{17d}}$\,\rangle$}\verb@@\\
\mbox{}\verb@@\hbox{$\langle\,$boundary and coboundary operators\nobreak\ {\footnotesize \NWlink{nuweb18a}{18a}}$\,\rangle$}\verb@@\\
\mbox{}\verb@@\hbox{$\langle\,$product of cell complexes\nobreak\ {\footnotesize \NWlink{nuweb18b}{18b}}$\,\rangle$}\verb@@\\
\mbox{}\verb@@\hbox{$\langle\,$2-skeleton extraction\nobreak\ {\footnotesize \NWlink{nuweb18c}{18c}}$\,\rangle$}\verb@@\\
\mbox{}\verb@@\hbox{$\langle\,$1-skeleton extraction\nobreak\ {\footnotesize \NWlink{nuweb19a}{19a}}$\,\rangle$}\verb@@\\
\mbox{}\verb@@\hbox{$\langle\,$0-coboundary computation\nobreak\ {\footnotesize \NWlink{nuweb19b}{19b}}$\,\rangle$}\verb@@\\
\mbox{}\verb@@\hbox{$\langle\,$1-coboundary computation\nobreak\ {\footnotesize \NWlink{nuweb19c}{19c}}$\,\rangle$}\verb@@\\
\mbox{}\verb@@\hbox{$\langle\,$2-coboundary computation\nobreak\ {\footnotesize \NWlink{nuweb20a}{20a}}$\,\rangle$}\verb@@\\
\mbox{}\verb@@\hbox{$\langle\,$boundary chain visualisation\nobreak\ {\footnotesize \NWlink{nuweb20b}{20b}}$\,\rangle$}\verb@@\\
\mbox{}\verb@@{\NWsep}
\end{list}
\vspace{-2ex}
\end{minipage}\\[4ex]
\end{flushleft}
%-------------------------------------------------------------------------------

%-------------------------------------------------------------------------------
\begin{flushleft} \small
\begin{minipage}{\linewidth} \label{scrap33}
\protect\makebox[0ex][r]{\NWtarget{nuweb17b}{\rule{0ex}{0ex}}\hspace{1em}}$\langle\,$input of 2D topology and geometry data\nobreak\ {\footnotesize 17b}$\,\rangle\equiv$
\vspace{-1ex}
\begin{list}{}{} \item
\mbox{}\verb@@\\
\mbox{}\verb@# input of topology and geometry@\\
\mbox{}\verb@V2 = [[4,10],[8,10],[14,10],[8,7],[14,7],[4,4],[8,4],[14,4]]@\\
\mbox{}\verb@EV = [[0,1],[1,2],[3,4],[5,6],[6,7],[0,5],[1,3],[2,4],[3,6],[4,7]]@\\
\mbox{}\verb@FV = [[0,1,3,5,6],[1,2,3,4],[3,4,6,7]]@\\
\mbox{}\verb@@{\NWsep}
\end{list}
\vspace{-1ex}
\footnotesize\addtolength{\baselineskip}{-1ex}
\begin{list}{}{\setlength{\itemsep}{-\parsep}\setlength{\itemindent}{-\leftmargin}}
\item \NWtxtMacroRefIn\ \NWlink{nuweb17a}{17a}.
\end{list}
\end{minipage}\\[4ex]
\end{flushleft}
%-------------------------------------------------------------------------------

%-------------------------------------------------------------------------------
\begin{flushleft} \small
\begin{minipage}{\linewidth} \label{scrap34}
\protect\makebox[0ex][r]{\NWtarget{nuweb17c}{\rule{0ex}{0ex}}\hspace{1em}}$\langle\,$characteristic matrices\nobreak\ {\footnotesize 17c}$\,\rangle\equiv$
\vspace{-1ex}
\begin{list}{}{} \item
\mbox{}\verb@# characteristic matrices@\\
\mbox{}\verb@csrFV = csrCreate(FV)@\\
\mbox{}\verb@csrEV = csrCreate(EV)@\\
\mbox{}\verb@print "\nFV =\n", csrToMatrixRepresentation(csrFV)@\\
\mbox{}\verb@print "\nEV =\n", csrToMatrixRepresentation(csrEV)@\\
\mbox{}\verb@@{\NWsep}
\end{list}
\vspace{-1ex}
\footnotesize\addtolength{\baselineskip}{-1ex}
\begin{list}{}{\setlength{\itemsep}{-\parsep}\setlength{\itemindent}{-\leftmargin}}
\item \NWtxtMacroRefIn\ \NWlink{nuweb17a}{17a}.
\end{list}
\end{minipage}\\[4ex]
\end{flushleft}
%-------------------------------------------------------------------------------

%-------------------------------------------------------------------------------
\begin{flushleft} \small
\begin{minipage}{\linewidth} \label{scrap35}
\protect\makebox[0ex][r]{\NWtarget{nuweb17d}{\rule{0ex}{0ex}}\hspace{1em}}$\langle\,$incidence matrix\nobreak\ {\footnotesize 17d}$\,\rangle\equiv$
\vspace{-1ex}
\begin{list}{}{} \item
\mbox{}\verb@# product@\\
\mbox{}\verb@csrEF = matrixProduct(csrEV, csrTranspose(csrFV))@\\
\mbox{}\verb@print "\nEF =\n", csrToMatrixRepresentation(csrEF)@\\
\mbox{}\verb@@{\NWsep}
\end{list}
\vspace{-1ex}
\footnotesize\addtolength{\baselineskip}{-1ex}
\begin{list}{}{\setlength{\itemsep}{-\parsep}\setlength{\itemindent}{-\leftmargin}}
\item \NWtxtMacroRefIn\ \NWlink{nuweb17a}{17a}.
\end{list}
\end{minipage}\\[4ex]
\end{flushleft}
%-------------------------------------------------------------------------------

%-------------------------------------------------------------------------------
\begin{flushleft} \small
\begin{minipage}{\linewidth} \label{scrap36}
\protect\makebox[0ex][r]{\NWtarget{nuweb18a}{\rule{0ex}{0ex}}\hspace{1em}}$\langle\,$boundary and coboundary operators\nobreak\ {\footnotesize 18a}$\,\rangle\equiv$
\vspace{-1ex}
\begin{list}{}{} \item
\mbox{}\verb@# boundary and coboundary operators@\\
\mbox{}\verb@facetLengths = [csrCell.getnnz() for csrCell in csrEV]@\\
\mbox{}\verb@boundary = csrBoundaryFilter(csrEF,facetLengths)@\\
\mbox{}\verb@coboundary_1 = csrTranspose(boundary)@\\
\mbox{}\verb@print "\ncoboundary_1 =\n", csrToMatrixRepresentation(coboundary_1)@\\
\mbox{}\verb@@{\NWsep}
\end{list}
\vspace{-1ex}
\footnotesize\addtolength{\baselineskip}{-1ex}
\begin{list}{}{\setlength{\itemsep}{-\parsep}\setlength{\itemindent}{-\leftmargin}}
\item \NWtxtMacroRefIn\ \NWlink{nuweb17a}{17a}.
\end{list}
\end{minipage}\\[4ex]
\end{flushleft}
%-------------------------------------------------------------------------------

%-------------------------------------------------------------------------------
\begin{flushleft} \small
\begin{minipage}{\linewidth} \label{scrap37}
\protect\makebox[0ex][r]{\NWtarget{nuweb18b}{\rule{0ex}{0ex}}\hspace{1em}}$\langle\,$product of cell complexes\nobreak\ {\footnotesize 18b}$\,\rangle\equiv$
\vspace{-1ex}
\begin{list}{}{} \item
\mbox{}\verb@# product operator@\\
\mbox{}\verb@mod_2D = (V2,FV)@\\
\mbox{}\verb@V1,topol_0 = [[0.],[1.],[2.]], [[0],[1],[2]]@\\
\mbox{}\verb@topol_1 = [[0,1],[1,2]]@\\
\mbox{}\verb@mod_0D = (V1,topol_0)@\\
\mbox{}\verb@mod_1D = (V1,topol_1)@\\
\mbox{}\verb@V3,CV = larModelProduct([mod_2D,mod_1D])@\\
\mbox{}\verb@mod_3D = (V3,CV)@\\
\mbox{}\verb@VIEW(EXPLODE(1.2,1.2,1.2)(MKPOLS(mod_3D)))@\\
\mbox{}\verb@print "\nk_3 =", len(CV), "\n"@\\
\mbox{}\verb@@{\NWsep}
\end{list}
\vspace{-1ex}
\footnotesize\addtolength{\baselineskip}{-1ex}
\begin{list}{}{\setlength{\itemsep}{-\parsep}\setlength{\itemindent}{-\leftmargin}}
\item \NWtxtMacroRefIn\ \NWlink{nuweb17a}{17a}.
\end{list}
\end{minipage}\\[4ex]
\end{flushleft}
%-------------------------------------------------------------------------------

%-------------------------------------------------------------------------------
\begin{flushleft} \small
\begin{minipage}{\linewidth} \label{scrap38}
\protect\makebox[0ex][r]{\NWtarget{nuweb18c}{\rule{0ex}{0ex}}\hspace{1em}}$\langle\,$2-skeleton extraction\nobreak\ {\footnotesize 18c}$\,\rangle\equiv$
\vspace{-1ex}
\begin{list}{}{} \item
\mbox{}\verb@# 2-skeleton of the 3D product complex@\\
\mbox{}\verb@mod_2D_1 = (V2,EV)@\\
\mbox{}\verb@mod_3D_h2 = larModelProduct([mod_2D,mod_0D])@\\
\mbox{}\verb@mod_3D_v2 = larModelProduct([mod_2D_1,mod_1D])@\\
\mbox{}\verb@_,FV_h = mod_3D_h2@\\
\mbox{}\verb@_,FV_v = mod_3D_v2@\\
\mbox{}\verb@FV3 = FV_h + FV_v@\\
\mbox{}\verb@SK2 = (V3,FV3)@\\
\mbox{}\verb@VIEW(EXPLODE(1.2,1.2,1.2)(MKPOLS(SK2)))@\\
\mbox{}\verb@print "\nk_2 =", len(FV3), "\n"@\\
\mbox{}\verb@@{\NWsep}
\end{list}
\vspace{-1ex}
\footnotesize\addtolength{\baselineskip}{-1ex}
\begin{list}{}{\setlength{\itemsep}{-\parsep}\setlength{\itemindent}{-\leftmargin}}
\item \NWtxtMacroRefIn\ \NWlink{nuweb17a}{17a}.
\end{list}
\end{minipage}\\[4ex]
\end{flushleft}
%-------------------------------------------------------------------------------

%-------------------------------------------------------------------------------
\begin{flushleft} \small
\begin{minipage}{\linewidth} \label{scrap39}
\protect\makebox[0ex][r]{\NWtarget{nuweb19a}{\rule{0ex}{0ex}}\hspace{1em}}$\langle\,$1-skeleton extraction\nobreak\ {\footnotesize 19a}$\,\rangle\equiv$
\vspace{-1ex}
\begin{list}{}{} \item
\mbox{}\verb@# 1-skeleton of the 3D product complex @\\
\mbox{}\verb@mod_2D_0 = (V2,AA(LIST)(range(len(V2))))@\\
\mbox{}\verb@mod_3D_h1 = larModelProduct([mod_2D_1,mod_0D])@\\
\mbox{}\verb@mod_3D_v1 = larModelProduct([mod_2D_0,mod_1D])@\\
\mbox{}\verb@_,EV_h = mod_3D_h1@\\
\mbox{}\verb@_,EV_v = mod_3D_v1@\\
\mbox{}\verb@EV3 = EV_h + EV_v@\\
\mbox{}\verb@SK1 = (V3,EV3)@\\
\mbox{}\verb@VIEW(EXPLODE(1.2,1.2,1.2)(MKPOLS(SK1)))@\\
\mbox{}\verb@print "\nk_1 =", len(EV3), "\n"@\\
\mbox{}\verb@@{\NWsep}
\end{list}
\vspace{-1ex}
\footnotesize\addtolength{\baselineskip}{-1ex}
\begin{list}{}{\setlength{\itemsep}{-\parsep}\setlength{\itemindent}{-\leftmargin}}
\item \NWtxtMacroRefIn\ \NWlink{nuweb17a}{17a}.
\end{list}
\end{minipage}\\[4ex]
\end{flushleft}
%-------------------------------------------------------------------------------

%-------------------------------------------------------------------------------
\begin{flushleft} \small
\begin{minipage}{\linewidth} \label{scrap40}
\protect\makebox[0ex][r]{\NWtarget{nuweb19b}{\rule{0ex}{0ex}}\hspace{1em}}$\langle\,$0-coboundary computation\nobreak\ {\footnotesize 19b}$\,\rangle\equiv$
\vspace{-1ex}
\begin{list}{}{} \item
\mbox{}\verb@# boundary and coboundary operators@\\
\mbox{}\verb@np.set_printoptions(threshold=sys.maxint)@\\
\mbox{}\verb@csrFV3 = csrCreate(FV3)@\\
\mbox{}\verb@csrEV3 = csrCreate(EV3)@\\
\mbox{}\verb@csrVE3 = csrTranspose(csrEV3)@\\
\mbox{}\verb@facetLengths = [csrCell.getnnz() for csrCell in csrEV3]@\\
\mbox{}\verb@boundary = csrBoundaryFilter(csrVE3,facetLengths)@\\
\mbox{}\verb@coboundary_0 = csrTranspose(boundary)@\\
\mbox{}\verb@print "\ncoboundary_0 =\n", csrToMatrixRepresentation(coboundary_0)@\\
\mbox{}\verb@@{\NWsep}
\end{list}
\vspace{-1ex}
\footnotesize\addtolength{\baselineskip}{-1ex}
\begin{list}{}{\setlength{\itemsep}{-\parsep}\setlength{\itemindent}{-\leftmargin}}
\item \NWtxtMacroRefIn\ \NWlink{nuweb17a}{17a}.
\end{list}
\end{minipage}\\[4ex]
\end{flushleft}
%-------------------------------------------------------------------------------

%-------------------------------------------------------------------------------
\begin{flushleft} \small
\begin{minipage}{\linewidth} \label{scrap41}
\protect\makebox[0ex][r]{\NWtarget{nuweb19c}{\rule{0ex}{0ex}}\hspace{1em}}$\langle\,$1-coboundary computation\nobreak\ {\footnotesize 19c}$\,\rangle\equiv$
\vspace{-1ex}
\begin{list}{}{} \item
\mbox{}\verb@csrEF3 = matrixProduct(csrEV3, csrTranspose(csrFV3))@\\
\mbox{}\verb@facetLengths = [csrCell.getnnz() for csrCell in csrFV3]@\\
\mbox{}\verb@boundary = csrBoundaryFilter(csrEF3,facetLengths)@\\
\mbox{}\verb@coboundary_1 = csrTranspose(boundary)@\\
\mbox{}\verb@print "\ncoboundary_1.T =\n", csrToMatrixRepresentation(coboundary_1.T)@\\
\mbox{}\verb@@{\NWsep}
\end{list}
\vspace{-1ex}
\footnotesize\addtolength{\baselineskip}{-1ex}
\begin{list}{}{\setlength{\itemsep}{-\parsep}\setlength{\itemindent}{-\leftmargin}}
\item \NWtxtMacroRefIn\ \NWlink{nuweb17a}{17a}.
\end{list}
\end{minipage}\\[4ex]
\end{flushleft}
%-------------------------------------------------------------------------------

%-------------------------------------------------------------------------------
\begin{flushleft} \small
\begin{minipage}{\linewidth} \label{scrap42}
\protect\makebox[0ex][r]{\NWtarget{nuweb20a}{\rule{0ex}{0ex}}\hspace{1em}}$\langle\,$2-coboundary computation\nobreak\ {\footnotesize 20a}$\,\rangle\equiv$
\vspace{-1ex}
\begin{list}{}{} \item
\mbox{}\verb@csrCV = csrCreate(CV)@\\
\mbox{}\verb@csrFC3 = matrixProduct(csrFV3, csrTranspose(csrCV))@\\
\mbox{}\verb@facetLengths = [csrCell.getnnz() for csrCell in csrCV]@\\
\mbox{}\verb@boundary = csrBoundaryFilter(csrFC3,facetLengths)@\\
\mbox{}\verb@coboundary_2 = csrTranspose(boundary)@\\
\mbox{}\verb@print "\ncoboundary_2 =\n", csrToMatrixRepresentation(coboundary_2)@\\
\mbox{}\verb@@{\NWsep}
\end{list}
\vspace{-1ex}
\footnotesize\addtolength{\baselineskip}{-1ex}
\begin{list}{}{\setlength{\itemsep}{-\parsep}\setlength{\itemindent}{-\leftmargin}}
\item \NWtxtMacroRefIn\ \NWlink{nuweb17a}{17a}.
\end{list}
\end{minipage}\\[4ex]
\end{flushleft}
%-------------------------------------------------------------------------------

%-------------------------------------------------------------------------------
\begin{flushleft} \small
\begin{minipage}{\linewidth} \label{scrap43}
\protect\makebox[0ex][r]{\NWtarget{nuweb20b}{\rule{0ex}{0ex}}\hspace{1em}}$\langle\,$boundary chain visualisation\nobreak\ {\footnotesize 20b}$\,\rangle\equiv$
\vspace{-1ex}
\begin{list}{}{} \item
\mbox{}\verb@# boundary chain visualisation@\\
\mbox{}\verb@boundaryCells_2 = boundaryCells(CV,FV3)@\\
\mbox{}\verb@boundary = (V3,[FV3[k] for k in boundaryCells_2])@\\
\mbox{}\verb@VIEW(EXPLODE(1.5,1.5,1.5)(MKPOLS(boundary)))@\\
\mbox{}\verb@@{\NWsep}
\end{list}
\vspace{-1ex}
\footnotesize\addtolength{\baselineskip}{-1ex}
\begin{list}{}{\setlength{\itemsep}{-\parsep}\setlength{\itemindent}{-\leftmargin}}
\item \NWtxtMacroRefIn\ \NWlink{nuweb17a}{17a}.
\end{list}
\end{minipage}\\[4ex]
\end{flushleft}
%-------------------------------------------------------------------------------



\subsection{Boundary of 3D simplicial grid}

%-------------------------------------------------------------------------------
\begin{flushleft} \small
\begin{minipage}{\linewidth} \label{scrap44}
\protect\makebox[0ex][r]{\NWtarget{nuweb20c}{\rule{0ex}{0ex}}\hspace{1em}}\verb@"test/py/larcc/ex2.py"@\nobreak\ {\footnotesize 20c }$\equiv$
\vspace{-1ex}
\begin{list}{}{} \item
\mbox{}\verb@@\\
\mbox{}\verb@@\hbox{$\langle\,$boundary of 3D simplicial grid\nobreak\ {\footnotesize \NWlink{nuweb20d}{20d}}$\,\rangle$}\verb@@\\
\mbox{}\verb@@{\NWsep}
\end{list}
\vspace{-2ex}
\end{minipage}\\[4ex]
\end{flushleft}
%-------------------------------------------------------------------------------

%-------------------------------------------------------------------------------
\begin{flushleft} \small
\begin{minipage}{\linewidth} \label{scrap45}
\protect\makebox[0ex][r]{\NWtarget{nuweb20d}{\rule{0ex}{0ex}}\hspace{1em}}$\langle\,$boundary of 3D simplicial grid\nobreak\ {\footnotesize 20d}$\,\rangle\equiv$
\vspace{-1ex}
\begin{list}{}{} \item
\mbox{}\verb@from simplexn import *@\\
\mbox{}\verb@from larcc import *@\\
\mbox{}\verb@@\\
\mbox{}\verb@V,CV = larSimplexGrid([10,10,3])@\\
\mbox{}\verb@VIEW(EXPLODE(1.5,1.5,1.5)(MKPOLS((V,CV))))@\\
\mbox{}\verb@SK2 = (V,larSimplexFacets(CV))@\\
\mbox{}\verb@VIEW(EXPLODE(1.5,1.5,1.5)(MKPOLS(SK2)))@\\
\mbox{}\verb@_,FV = SK2@\\
\mbox{}\verb@SK1 = (V,larSimplexFacets(FV))@\\
\mbox{}\verb@_,EV = SK1@\\
\mbox{}\verb@VIEW(EXPLODE(1.5,1.5,1.5)(MKPOLS(SK1)))@\\
\mbox{}\verb@@\\
\mbox{}\verb@boundaryCells_2 = boundaryCells(CV,FV)@\\
\mbox{}\verb@boundary = (V,[FV[k] for k in boundaryCells_2])@\\
\mbox{}\verb@VIEW(EXPLODE(1.5,1.5,1.5)(MKPOLS(boundary)))@\\
\mbox{}\verb@print "\nboundaryCells_2 =\n", boundaryCells_2@\\
\mbox{}\verb@@{\NWsep}
\end{list}
\vspace{-1ex}
\footnotesize\addtolength{\baselineskip}{-1ex}
\begin{list}{}{\setlength{\itemsep}{-\parsep}\setlength{\itemindent}{-\leftmargin}}
\item \NWtxtMacroRefIn\ \NWlink{nuweb20c}{20c}.
\end{list}
\end{minipage}\\[4ex]
\end{flushleft}
%-------------------------------------------------------------------------------


\subsection{Oriented boundary of a random simplicial complex}


%-------------------------------------------------------------------------------
\begin{flushleft} \small
\begin{minipage}{\linewidth} \label{scrap46}
\protect\makebox[0ex][r]{\NWtarget{nuweb21a}{\rule{0ex}{0ex}}\hspace{1em}}\verb@"test/py/larcc/ex3.py"@\nobreak\ {\footnotesize 21a }$\equiv$
\vspace{-1ex}
\begin{list}{}{} \item
\mbox{}\verb@@\hbox{$\langle\,$Importing external modules\nobreak\ {\footnotesize \NWlink{nuweb21b}{21b}}$\,\rangle$}\verb@@\\
\mbox{}\verb@@\hbox{$\langle\,$Generating and viewing a random 3D simplicial complex\nobreak\ {\footnotesize \NWlink{nuweb21c}{21c}}$\,\rangle$}\verb@@\\
\mbox{}\verb@@\hbox{$\langle\,$Computing and viewing its non-oriented boundary\nobreak\ {\footnotesize \NWlink{nuweb21d}{21d}}$\,\rangle$}\verb@@\\
\mbox{}\verb@@\hbox{$\langle\,$Computing and viewing its oriented boundary\nobreak\ {\footnotesize \NWlink{nuweb22a}{22a}}$\,\rangle$}\verb@@\\
\mbox{}\verb@@{\NWsep}
\end{list}
\vspace{-2ex}
\end{minipage}\\[4ex]
\end{flushleft}
%-------------------------------------------------------------------------------


%-------------------------------------------------------------------------------
\begin{flushleft} \small
\begin{minipage}{\linewidth} \label{scrap47}
\protect\makebox[0ex][r]{\NWtarget{nuweb21b}{\rule{0ex}{0ex}}\hspace{1em}}$\langle\,$Importing external modules\nobreak\ {\footnotesize 21b}$\,\rangle\equiv$
\vspace{-1ex}
\begin{list}{}{} \item
\mbox{}\verb@from simplexn import *@\\
\mbox{}\verb@from larcc import *@\\
\mbox{}\verb@from scipy.spatial import Delaunay@\\
\mbox{}\verb@import numpy as np@\\
\mbox{}\verb@@{\NWsep}
\end{list}
\vspace{-1ex}
\footnotesize\addtolength{\baselineskip}{-1ex}
\begin{list}{}{\setlength{\itemsep}{-\parsep}\setlength{\itemindent}{-\leftmargin}}
\item \NWtxtMacroRefIn\ \NWlink{nuweb21a}{21a}.
\end{list}
\end{minipage}\\[4ex]
\end{flushleft}
%-------------------------------------------------------------------------------

%-------------------------------------------------------------------------------
\begin{flushleft} \small
\begin{minipage}{\linewidth} \label{scrap48}
\protect\makebox[0ex][r]{\NWtarget{nuweb21c}{\rule{0ex}{0ex}}\hspace{1em}}$\langle\,$Generating and viewing a random 3D simplicial complex\nobreak\ {\footnotesize 21c}$\,\rangle\equiv$
\vspace{-1ex}
\begin{list}{}{} \item
\mbox{}\verb@verts = np.random.rand(10000, 3) # 1000 points in 3-d@\\
\mbox{}\verb@verts = [AA(lambda x: 2*x)(VECTDIFF([vert,[0.5,0.5,0.5]])) for vert in verts]@\\
\mbox{}\verb@verts = [vert for vert in verts if VECTNORM(vert) < 1.0]@\\
\mbox{}\verb@tetra = Delaunay(verts)@\\
\mbox{}\verb@cells = [cell for cell in tetra.vertices.tolist()@\\
\mbox{}\verb@         if  ((verts[cell[0]][2]<0) and (verts[cell[1]][2]<0) @\\
\mbox{}\verb@               and (verts[cell[2]][2]<0) and (verts[cell[3]][2]<0) ) ]@\\
\mbox{}\verb@V, CV = verts, cells@\\
\mbox{}\verb@VIEW(MKPOL([V,AA(AA(lambda k:k+1))(CV),[]]))@\\
\mbox{}\verb@@{\NWsep}
\end{list}
\vspace{-1ex}
\footnotesize\addtolength{\baselineskip}{-1ex}
\begin{list}{}{\setlength{\itemsep}{-\parsep}\setlength{\itemindent}{-\leftmargin}}
\item \NWtxtMacroRefIn\ \NWlink{nuweb21a}{21a}.
\end{list}
\end{minipage}\\[4ex]
\end{flushleft}
%-------------------------------------------------------------------------------

%-------------------------------------------------------------------------------
\begin{flushleft} \small
\begin{minipage}{\linewidth} \label{scrap49}
\protect\makebox[0ex][r]{\NWtarget{nuweb21d}{\rule{0ex}{0ex}}\hspace{1em}}$\langle\,$Computing and viewing its non-oriented boundary\nobreak\ {\footnotesize 21d}$\,\rangle\equiv$
\vspace{-1ex}
\begin{list}{}{} \item
\mbox{}\verb@FV = larSimplexFacets(CV)@\\
\mbox{}\verb@VIEW(MKPOL([V,AA(AA(lambda k:k+1))(FV),[]]))@\\
\mbox{}\verb@boundaryCells_2 = boundaryCells(CV,FV)@\\
\mbox{}\verb@print "\nboundaryCells_2 =\n", boundaryCells_2@\\
\mbox{}\verb@bndry = (V,[FV[k] for k in boundaryCells_2])@\\
\mbox{}\verb@VIEW(EXPLODE(1.5,1.5,1.5)(MKPOLS(bndry)))@\\
\mbox{}\verb@@{\NWsep}
\end{list}
\vspace{-1ex}
\footnotesize\addtolength{\baselineskip}{-1ex}
\begin{list}{}{\setlength{\itemsep}{-\parsep}\setlength{\itemindent}{-\leftmargin}}
\item \NWtxtMacroRefIn\ \NWlink{nuweb21a}{21a}.
\end{list}
\end{minipage}\\[4ex]
\end{flushleft}
%-------------------------------------------------------------------------------

%-------------------------------------------------------------------------------
\begin{flushleft} \small
\begin{minipage}{\linewidth} \label{scrap50}
\protect\makebox[0ex][r]{\NWtarget{nuweb22a}{\rule{0ex}{0ex}}\hspace{1em}}$\langle\,$Computing and viewing its oriented boundary\nobreak\ {\footnotesize 22a}$\,\rangle\equiv$
\vspace{-1ex}
\begin{list}{}{} \item
\mbox{}\verb@boundaryCells_2 = signedBoundaryCells(V,CV,FV)@\\
\mbox{}\verb@print "\nboundaryCells_2 =\n", boundaryCells_2@\\
\mbox{}\verb@def swap(mylist): return [mylist[1]]+[mylist[0]]+mylist[2:]@\\
\mbox{}\verb@boundaryFV = [FV[-k] if k<0 else swap(FV[k]) for k in boundaryCells_2]@\\
\mbox{}\verb@bndry = (V,boundaryFV)@\\
\mbox{}\verb@VIEW(EXPLODE(1.5,1.5,1.5)(MKPOLS(bndry)))@\\
\mbox{}\verb@@{\NWsep}
\end{list}
\vspace{-1ex}
\footnotesize\addtolength{\baselineskip}{-1ex}
\begin{list}{}{\setlength{\itemsep}{-\parsep}\setlength{\itemindent}{-\leftmargin}}
\item \NWtxtMacroRefIn\ \NWlink{nuweb21a}{21a}.
\end{list}
\end{minipage}\\[4ex]
\end{flushleft}
%-------------------------------------------------------------------------------

\subsection{Oriented boundary of a simplicial grid}

%-------------------------------------------------------------------------------
\begin{flushleft} \small
\begin{minipage}{\linewidth} \label{scrap51}
\protect\makebox[0ex][r]{\NWtarget{nuweb22b}{\rule{0ex}{0ex}}\hspace{1em}}\verb@"test/py/larcc/ex4.py"@\nobreak\ {\footnotesize 22b }$\equiv$
\vspace{-1ex}
\begin{list}{}{} \item
\mbox{}\verb@@\hbox{$\langle\,$Generate and view a 3D simplicial grid\nobreak\ {\footnotesize \NWlink{nuweb22c}{22c}}$\,\rangle$}\verb@@\\
\mbox{}\verb@@\hbox{$\langle\,$Computing and viewing the 2-skeleton of simplicial grid\nobreak\ {\footnotesize \NWlink{nuweb22d}{22d}}$\,\rangle$}\verb@@\\
\mbox{}\verb@@\hbox{$\langle\,$Computing and viewing the oriented boundary of simplicial grid\nobreak\ {\footnotesize \NWlink{nuweb22e}{22e}}$\,\rangle$}\verb@@\\
\mbox{}\verb@@{\NWsep}
\end{list}
\vspace{-2ex}
\end{minipage}\\[4ex]
\end{flushleft}
%-------------------------------------------------------------------------------


%-------------------------------------------------------------------------------
\begin{flushleft} \small
\begin{minipage}{\linewidth} \label{scrap52}
\protect\makebox[0ex][r]{\NWtarget{nuweb22c}{\rule{0ex}{0ex}}\hspace{1em}}$\langle\,$Generate and view a 3D simplicial grid\nobreak\ {\footnotesize 22c}$\,\rangle\equiv$
\vspace{-1ex}
\begin{list}{}{} \item
\mbox{}\verb@from simplexn import *@\\
\mbox{}\verb@from larcc import *@\\
\mbox{}\verb@V,CV = larSimplexGrid([4,4,4])@\\
\mbox{}\verb@VIEW(EXPLODE(1.5,1.5,1.5)(MKPOLS((V,CV))))@\\
\mbox{}\verb@@{\NWsep}
\end{list}
\vspace{-1ex}
\footnotesize\addtolength{\baselineskip}{-1ex}
\begin{list}{}{\setlength{\itemsep}{-\parsep}\setlength{\itemindent}{-\leftmargin}}
\item \NWtxtMacroRefIn\ \NWlink{nuweb22b}{22b}.
\end{list}
\end{minipage}\\[4ex]
\end{flushleft}
%-------------------------------------------------------------------------------

%-------------------------------------------------------------------------------
\begin{flushleft} \small
\begin{minipage}{\linewidth} \label{scrap53}
\protect\makebox[0ex][r]{\NWtarget{nuweb22d}{\rule{0ex}{0ex}}\hspace{1em}}$\langle\,$Computing and viewing the 2-skeleton of simplicial grid\nobreak\ {\footnotesize 22d}$\,\rangle\equiv$
\vspace{-1ex}
\begin{list}{}{} \item
\mbox{}\verb@FV = larSimplexFacets(CV)@\\
\mbox{}\verb@EV = larSimplexFacets(FV)@\\
\mbox{}\verb@VIEW(EXPLODE(1.5,1.5,1.5)(MKPOLS((V,FV))))@\\
\mbox{}\verb@@{\NWsep}
\end{list}
\vspace{-1ex}
\footnotesize\addtolength{\baselineskip}{-1ex}
\begin{list}{}{\setlength{\itemsep}{-\parsep}\setlength{\itemindent}{-\leftmargin}}
\item \NWtxtMacroRefIn\ \NWlink{nuweb22b}{22b}.
\end{list}
\end{minipage}\\[4ex]
\end{flushleft}
%-------------------------------------------------------------------------------

%-------------------------------------------------------------------------------
\begin{flushleft} \small
\begin{minipage}{\linewidth} \label{scrap54}
\protect\makebox[0ex][r]{\NWtarget{nuweb22e}{\rule{0ex}{0ex}}\hspace{1em}}$\langle\,$Computing and viewing the oriented boundary of simplicial grid\nobreak\ {\footnotesize 22e}$\,\rangle\equiv$
\vspace{-1ex}
\begin{list}{}{} \item
\mbox{}\verb@csrSignedBoundaryMat = signedBoundary (V,CV,FV)@\\
\mbox{}\verb@boundaryCells_2 = signedBoundaryCells(V,CV,FV)@\\
\mbox{}\verb@def swap(l): return [l[1],l[0],l[2]]@\\
\mbox{}\verb@boundaryFV = [FV[-k] if k<0 else swap(FV[k]) for k in boundaryCells_2]@\\
\mbox{}\verb@boundary = (V,boundaryFV)@\\
\mbox{}\verb@VIEW(EXPLODE(1.5,1.5,1.5)(MKPOLS(boundary)))@\\
\mbox{}\verb@@{\NWsep}
\end{list}
\vspace{-1ex}
\footnotesize\addtolength{\baselineskip}{-1ex}
\begin{list}{}{\setlength{\itemsep}{-\parsep}\setlength{\itemindent}{-\leftmargin}}
\item \NWtxtMacroRefIn\ \NWlink{nuweb22b}{22b}.
\end{list}
\end{minipage}\\[4ex]
\end{flushleft}
%-------------------------------------------------------------------------------


\subsection{Skeletons and oriented boundary of a simplicial complex}


%-------------------------------------------------------------------------------
\begin{flushleft} \small
\begin{minipage}{\linewidth} \label{scrap55}
\protect\makebox[0ex][r]{\NWtarget{nuweb23a}{\rule{0ex}{0ex}}\hspace{1em}}\verb@"test/py/larcc/ex5.py"@\nobreak\ {\footnotesize 23a }$\equiv$
\vspace{-1ex}
\begin{list}{}{} \item
\mbox{}\verb@@\hbox{$\langle\,$Skeletons computation and vilualisation\nobreak\ {\footnotesize \NWlink{nuweb23b}{23b}}$\,\rangle$}\verb@@\\
\mbox{}\verb@@\hbox{$\langle\,$Oriented boundary matrix visualization\nobreak\ {\footnotesize \NWlink{nuweb23c}{23c}}$\,\rangle$}\verb@@\\
\mbox{}\verb@@\hbox{$\langle\,$Computation of oriented boundary cells\nobreak\ {\footnotesize \NWlink{nuweb23d}{23d}}$\,\rangle$}\verb@@\\
\mbox{}\verb@@{\NWsep}
\end{list}
\vspace{-2ex}
\end{minipage}\\[4ex]
\end{flushleft}
%-------------------------------------------------------------------------------


%-------------------------------------------------------------------------------
\begin{flushleft} \small
\begin{minipage}{\linewidth} \label{scrap56}
\protect\makebox[0ex][r]{\NWtarget{nuweb23b}{\rule{0ex}{0ex}}\hspace{1em}}$\langle\,$Skeletons computation and vilualisation\nobreak\ {\footnotesize 23b}$\,\rangle\equiv$
\vspace{-1ex}
\begin{list}{}{} \item
\mbox{}\verb@from simplexn import *@\\
\mbox{}\verb@from larcc import *@\\
\mbox{}\verb@V,FV = larSimplexGrid([3,3])@\\
\mbox{}\verb@VIEW(EXPLODE(1.5,1.5,1.5)(MKPOLS((V,FV))))@\\
\mbox{}\verb@EV = larSimplexFacets(FV)@\\
\mbox{}\verb@VIEW(EXPLODE(1.5,1.5,1.5)(MKPOLS((V,EV))))@\\
\mbox{}\verb@VV = larSimplexFacets(EV)@\\
\mbox{}\verb@VIEW(EXPLODE(1.5,1.5,1.5)(MKPOLS((V,VV))))@\\
\mbox{}\verb@@{\NWsep}
\end{list}
\vspace{-1ex}
\footnotesize\addtolength{\baselineskip}{-1ex}
\begin{list}{}{\setlength{\itemsep}{-\parsep}\setlength{\itemindent}{-\leftmargin}}
\item \NWtxtMacroRefIn\ \NWlink{nuweb23a}{23a}.
\end{list}
\end{minipage}\\[4ex]
\end{flushleft}
%-------------------------------------------------------------------------------

%-------------------------------------------------------------------------------
\begin{flushleft} \small
\begin{minipage}{\linewidth} \label{scrap57}
\protect\makebox[0ex][r]{\NWtarget{nuweb23c}{\rule{0ex}{0ex}}\hspace{1em}}$\langle\,$Oriented boundary matrix visualization\nobreak\ {\footnotesize 23c}$\,\rangle\equiv$
\vspace{-1ex}
\begin{list}{}{} \item
\mbox{}\verb@np.set_printoptions(threshold='nan')@\\
\mbox{}\verb@csrSignedBoundaryMat = signedBoundary (V,FV,EV)@\\
\mbox{}\verb@Z = csrToMatrixRepresentation(csrSignedBoundaryMat)@\\
\mbox{}\verb@print "\ncsrSignedBoundaryMat =\n", Z@\\
\mbox{}\verb@from pylab import *@\\
\mbox{}\verb@matshow(Z)@\\
\mbox{}\verb@show()@\\
\mbox{}\verb@@{\NWsep}
\end{list}
\vspace{-1ex}
\footnotesize\addtolength{\baselineskip}{-1ex}
\begin{list}{}{\setlength{\itemsep}{-\parsep}\setlength{\itemindent}{-\leftmargin}}
\item \NWtxtMacroRefIn\ \NWlink{nuweb23a}{23a}.
\end{list}
\end{minipage}\\[4ex]
\end{flushleft}
%-------------------------------------------------------------------------------

%-------------------------------------------------------------------------------
\begin{flushleft} \small
\begin{minipage}{\linewidth} \label{scrap58}
\protect\makebox[0ex][r]{\NWtarget{nuweb23d}{\rule{0ex}{0ex}}\hspace{1em}}$\langle\,$Computation of oriented boundary cells\nobreak\ {\footnotesize 23d}$\,\rangle\equiv$
\vspace{-1ex}
\begin{list}{}{} \item
\mbox{}\verb@boundaryCells_1 = signedBoundaryCells(V,FV,EV)@\\
\mbox{}\verb@print "\nboundaryCells_1 =\n", boundaryCells_1@\\
\mbox{}\verb@def swap(mylist): return [mylist[1]]+[mylist[0]]+mylist[2:]@\\
\mbox{}\verb@boundaryEV = [EV[-k] if k<0 else swap(EV[k]) for k in boundaryCells_1]@\\
\mbox{}\verb@bndry = (V,boundaryEV)@\\
\mbox{}\verb@VIEW(EXPLODE(1.5,1.5,1.5)(MKPOLS(bndry)))@\\
\mbox{}\verb@@{\NWsep}
\end{list}
\vspace{-1ex}
\footnotesize\addtolength{\baselineskip}{-1ex}
\begin{list}{}{\setlength{\itemsep}{-\parsep}\setlength{\itemindent}{-\leftmargin}}
\item \NWtxtMacroRefIn\ \NWlink{nuweb23a}{23a}.
\end{list}
\end{minipage}\\[4ex]
\end{flushleft}
%-------------------------------------------------------------------------------

\subsection{Boundary of random 2D simplicial complex}

%-------------------------------------------------------------------------------
\begin{flushleft} \small
\begin{minipage}{\linewidth} \label{scrap59}
\protect\makebox[0ex][r]{\NWtarget{nuweb24a}{\rule{0ex}{0ex}}\hspace{1em}}\verb@"test/py/larcc/ex6.py"@\nobreak\ {\footnotesize 24a }$\equiv$
\vspace{-1ex}
\begin{list}{}{} \item
\mbox{}\verb@from simplexn import *@\\
\mbox{}\verb@from larcc import *@\\
\mbox{}\verb@from scipy.spatial import Delaunay@\\
\mbox{}\verb@@\hbox{$\langle\,$Test for quasi-equilateral triangles\nobreak\ {\footnotesize \NWlink{nuweb24b}{24b}}$\,\rangle$}\verb@@\\
\mbox{}\verb@@\hbox{$\langle\,$Generation and selection of random triangles\nobreak\ {\footnotesize \NWlink{nuweb25a}{25a}}$\,\rangle$}\verb@@\\
\mbox{}\verb@@\hbox{$\langle\,$Boundary computation and visualisation\nobreak\ {\footnotesize \NWlink{nuweb25b}{25b}}$\,\rangle$}\verb@@\\
\mbox{}\verb@@{\NWsep}
\end{list}
\vspace{-2ex}
\end{minipage}\\[4ex]
\end{flushleft}
%-------------------------------------------------------------------------------


\begin{figure}[htbp] %  figure placement: here, top, bottom, or page
   \centering
   \includegraphics[height=0.25\linewidth,width=0.32\linewidth]{images/tria0} 
   \includegraphics[height=0.25\linewidth,width=0.32\linewidth]{images/tria1} 
   \includegraphics[height=0.25\linewidth,width=0.32\linewidth]{images/tria2} 
   \caption{example caption}
   \label{fig:example}
\end{figure}

%-------------------------------------------------------------------------------
\begin{flushleft} \small
\begin{minipage}{\linewidth} \label{scrap60}
\protect\makebox[0ex][r]{\NWtarget{nuweb24b}{\rule{0ex}{0ex}}\hspace{1em}}$\langle\,$Test for quasi-equilateral triangles\nobreak\ {\footnotesize 24b}$\,\rangle\equiv$
\vspace{-1ex}
\begin{list}{}{} \item
\mbox{}\verb@def quasiEquilateral(tria):@\\
\mbox{}\verb@    a = VECTNORM(VECTDIFF(tria[0:2]))@\\
\mbox{}\verb@    b = VECTNORM(VECTDIFF(tria[1:3]))@\\
\mbox{}\verb@    c = VECTNORM(VECTDIFF([tria[0],tria[2]]))@\\
\mbox{}\verb@    m = max(a,b,c)@\\
\mbox{}\verb@    if m/a < 1.7 and m/b < 1.7 and m/c < 1.7: return True@\\
\mbox{}\verb@    else: return False@\\
\mbox{}\verb@@{\NWsep}
\end{list}
\vspace{-1ex}
\footnotesize\addtolength{\baselineskip}{-1ex}
\begin{list}{}{\setlength{\itemsep}{-\parsep}\setlength{\itemindent}{-\leftmargin}}
\item \NWtxtMacroRefIn\ \NWlink{nuweb24a}{24a}.
\end{list}
\end{minipage}\\[4ex]
\end{flushleft}
%-------------------------------------------------------------------------------

%-------------------------------------------------------------------------------
\begin{flushleft} \small
\begin{minipage}{\linewidth} \label{scrap61}
\protect\makebox[0ex][r]{\NWtarget{nuweb25a}{\rule{0ex}{0ex}}\hspace{1em}}$\langle\,$Generation and selection of random triangles\nobreak\ {\footnotesize 25a}$\,\rangle\equiv$
\vspace{-1ex}
\begin{list}{}{} \item
\mbox{}\verb@verts = np.random.rand(20,2)@\\
\mbox{}\verb@verts = (verts - [0.5,0.5]) * 2@\\
\mbox{}\verb@triangles = Delaunay(verts)@\\
\mbox{}\verb@cells = [ cell for cell in triangles.vertices.tolist()@\\
\mbox{}\verb@         if (not quasiEquilateral([verts[k] for k in cell])) ]@\\
\mbox{}\verb@V, FV = AA(list)(verts), cells@\\
\mbox{}\verb@EV = larSimplexFacets(FV)@\\
\mbox{}\verb@pols2D = MKPOLS((V,FV))@\\
\mbox{}\verb@VIEW(EXPLODE(1.5,1.5,1.5)(pols2D))@\\
\mbox{}\verb@@{\NWsep}
\end{list}
\vspace{-1ex}
\footnotesize\addtolength{\baselineskip}{-1ex}
\begin{list}{}{\setlength{\itemsep}{-\parsep}\setlength{\itemindent}{-\leftmargin}}
\item \NWtxtMacroRefIn\ \NWlink{nuweb24a}{24a}.
\end{list}
\end{minipage}\\[4ex]
\end{flushleft}
%-------------------------------------------------------------------------------

%-------------------------------------------------------------------------------
\begin{flushleft} \small
\begin{minipage}{\linewidth} \label{scrap62}
\protect\makebox[0ex][r]{\NWtarget{nuweb25b}{\rule{0ex}{0ex}}\hspace{1em}}$\langle\,$Boundary computation and visualisation\nobreak\ {\footnotesize 25b}$\,\rangle\equiv$
\vspace{-1ex}
\begin{list}{}{} \item
\mbox{}\verb@boundaryCells_1 = signedBoundaryCells(V,FV,EV)@\\
\mbox{}\verb@print "\nboundaryCells_1 =\n", boundaryCells_1@\\
\mbox{}\verb@def swap(mylist): return [mylist[1]]+[mylist[0]]+mylist[2:]@\\
\mbox{}\verb@boundaryEV = [EV[-k] if k<0 else swap(EV[k]) for k in boundaryCells_1]@\\
\mbox{}\verb@bndry = (V,boundaryEV)@\\
\mbox{}\verb@VIEW(STRUCT(MKPOLS(bndry) + pols2D))@\\
\mbox{}\verb@VIEW(COLOR(RED)(STRUCT(MKPOLS(bndry))))@\\
\mbox{}\verb@@{\NWsep}
\end{list}
\vspace{-1ex}
\footnotesize\addtolength{\baselineskip}{-1ex}
\begin{list}{}{\setlength{\itemsep}{-\parsep}\setlength{\itemindent}{-\leftmargin}}
\item \NWtxtMacroRefIn\ \NWlink{nuweb24a}{24a}.
\end{list}
\end{minipage}\\[4ex]
\end{flushleft}
%-------------------------------------------------------------------------------

%-------------------------------------------------------------------------------
\begin{flushleft} \small
\begin{minipage}{\linewidth} \label{scrap63}
\protect\makebox[0ex][r]{\NWtarget{nuweb25c}{\rule{0ex}{0ex}}\hspace{1em}}$\langle\,$Compute the topologically ordered chain of boundary vertices\nobreak\ {\footnotesize 25c}$\,\rangle\equiv$
\vspace{-1ex}
\begin{list}{}{} \item
\mbox{}\verb@@\\
\mbox{}\verb@@{\NWsep}
\end{list}
\vspace{-1ex}
\footnotesize\addtolength{\baselineskip}{-1ex}
\begin{list}{}{\setlength{\itemsep}{-\parsep}\setlength{\itemindent}{-\leftmargin}}
\item {\NWtxtMacroNoRef}.
\end{list}
\end{minipage}\\[4ex]
\end{flushleft}
%-------------------------------------------------------------------------------

%-------------------------------------------------------------------------------
\begin{flushleft} \small
\begin{minipage}{\linewidth} \label{scrap64}
\protect\makebox[0ex][r]{\NWtarget{nuweb26a}{\rule{0ex}{0ex}}\hspace{1em}}$\langle\,$Decompose a permutation into cycles\nobreak\ {\footnotesize 26a}$\,\rangle\equiv$
\vspace{-1ex}
\begin{list}{}{} \item
\mbox{}\verb@def permutationOrbits(List):@\\
\mbox{}\verb@   d = dict((i,int(x)) for i,x in enumerate(List))@\\
\mbox{}\verb@   out = []@\\
\mbox{}\verb@   while d:@\\
\mbox{}\verb@      x = list(d)[0]@\\
\mbox{}\verb@      orbit = []@\\
\mbox{}\verb@      while x in d:@\\
\mbox{}\verb@         orbit += [x],@\\
\mbox{}\verb@         x = d.pop(x)@\\
\mbox{}\verb@      out += [CAT(orbit)+orbit[0]]@\\
\mbox{}\verb@   return out@\\
\mbox{}\verb@      @\\
\mbox{}\verb@if __name__ == "__main__":@\\
\mbox{}\verb@   print [2, 3, 4, 5, 6, 7, 0, 1]@\\
\mbox{}\verb@   print permutationOrbits([2, 3, 4, 5, 6, 7, 0, 1])@\\
\mbox{}\verb@   print [3,9,8,4,10,7,2,11,6,0,1,5]@\\
\mbox{}\verb@   print permutationOrbits([3,9,8,4,10,7,2,11,6,0,1,5])@\\
\mbox{}\verb@@{\NWsep}
\end{list}
\vspace{-1ex}
\footnotesize\addtolength{\baselineskip}{-1ex}
\begin{list}{}{\setlength{\itemsep}{-\parsep}\setlength{\itemindent}{-\leftmargin}}
\item {\NWtxtMacroNoRef}.
\end{list}
\end{minipage}\\[4ex]
\end{flushleft}
%-------------------------------------------------------------------------------

\subsection{Assemblies of simplices and hypercubes}

%-------------------------------------------------------------------------------
\begin{flushleft} \small
\begin{minipage}{\linewidth} \label{scrap65}
\protect\makebox[0ex][r]{\NWtarget{nuweb26b}{\rule{0ex}{0ex}}\hspace{1em}}\verb@"test/py/larcc/ex7.py"@\nobreak\ {\footnotesize 26b }$\equiv$
\vspace{-1ex}
\begin{list}{}{} \item
\mbox{}\verb@from simplexn import *@\\
\mbox{}\verb@from larcc import *@\\
\mbox{}\verb@from largrid import *@\\
\mbox{}\verb@@\hbox{$\langle\,$Definition of 1-dimensional LAR models\nobreak\ {\footnotesize \NWlink{nuweb27a}{27a}}$\,\rangle$}\verb@@\\
\mbox{}\verb@@\hbox{$\langle\,$Assembly generation of squares and triangles\nobreak\ {\footnotesize \NWlink{nuweb27b}{27b}}$\,\rangle$}\verb@@\\
\mbox{}\verb@@\hbox{$\langle\,$Assembly generation of cubes and tetrahedra\nobreak\ {\footnotesize \NWlink{nuweb27c}{27c}}$\,\rangle$}\verb@@\\
\mbox{}\verb@@{\NWsep}
\end{list}
\vspace{-2ex}
\end{minipage}\\[4ex]
\end{flushleft}
%-------------------------------------------------------------------------------

\begin{figure}[htbp] %  figure placement: here, top, bottom, or page
   \centering
   \includegraphics[width=0.405\linewidth]{images/assembly1} 
   \includegraphics[width=0.315\linewidth]{images/assembly2} 
   \caption{(a) Assemblies of squares and triangles; (b) assembly of cubes and tetrahedra.}
   \label{fig:example}
\end{figure}

%-------------------------------------------------------------------------------
\begin{flushleft} \small
\begin{minipage}{\linewidth} \label{scrap66}
\protect\makebox[0ex][r]{\NWtarget{nuweb27a}{\rule{0ex}{0ex}}\hspace{1em}}$\langle\,$Definition of 1-dimensional LAR models\nobreak\ {\footnotesize 27a}$\,\rangle\equiv$
\vspace{-1ex}
\begin{list}{}{} \item
\mbox{}\verb@geom_0,topol_0 = [[0.],[1.],[2.],[3.],[4.]],[[0,1],[1,2],[3,4]]@\\
\mbox{}\verb@geom_1,topol_1 = [[0.],[1.],[2.]], [[0,1],[1,2]]@\\
\mbox{}\verb@mod_0 = (geom_0,topol_0)@\\
\mbox{}\verb@mod_1 = (geom_1,topol_1)@\\
\mbox{}\verb@@{\NWsep}
\end{list}
\vspace{-1ex}
\footnotesize\addtolength{\baselineskip}{-1ex}
\begin{list}{}{\setlength{\itemsep}{-\parsep}\setlength{\itemindent}{-\leftmargin}}
\item \NWtxtMacroRefIn\ \NWlink{nuweb26b}{26b}.
\end{list}
\end{minipage}\\[4ex]
\end{flushleft}
%-------------------------------------------------------------------------------

%-------------------------------------------------------------------------------
\begin{flushleft} \small
\begin{minipage}{\linewidth} \label{scrap67}
\protect\makebox[0ex][r]{\NWtarget{nuweb27b}{\rule{0ex}{0ex}}\hspace{1em}}$\langle\,$Assembly generation of squares and triangles\nobreak\ {\footnotesize 27b}$\,\rangle\equiv$
\vspace{-1ex}
\begin{list}{}{} \item
\mbox{}\verb@squares = larModelProduct([mod_0,mod_1])@\\
\mbox{}\verb@V,FV = squares@\\
\mbox{}\verb@simplices = pivotSimplices(V,FV,d=2)@\\
\mbox{}\verb@VIEW(STRUCT([ MKPOL([V,AA(AA(C(SUM)(1)))(simplices),[]]),@\\
\mbox{}\verb@              SKEL_1(STRUCT(MKPOLS((V,FV)))) ]))@\\
\mbox{}\verb@@{\NWsep}
\end{list}
\vspace{-1ex}
\footnotesize\addtolength{\baselineskip}{-1ex}
\begin{list}{}{\setlength{\itemsep}{-\parsep}\setlength{\itemindent}{-\leftmargin}}
\item \NWtxtMacroRefIn\ \NWlink{nuweb26b}{26b}.
\end{list}
\end{minipage}\\[4ex]
\end{flushleft}
%-------------------------------------------------------------------------------

%-------------------------------------------------------------------------------
\begin{flushleft} \small
\begin{minipage}{\linewidth} \label{scrap68}
\protect\makebox[0ex][r]{\NWtarget{nuweb27c}{\rule{0ex}{0ex}}\hspace{1em}}$\langle\,$Assembly generation of cubes and tetrahedra\nobreak\ {\footnotesize 27c}$\,\rangle\equiv$
\vspace{-1ex}
\begin{list}{}{} \item
\mbox{}\verb@cubes = larModelProduct([squares,mod_0])@\\
\mbox{}\verb@V,CV = cubes@\\
\mbox{}\verb@simplices = pivotSimplices(V,CV,d=3)@\\
\mbox{}\verb@VIEW(STRUCT([ MKPOL([V,AA(AA(C(SUM)(1)))(simplices),[]]),@\\
\mbox{}\verb@           SKEL_1(STRUCT(MKPOLS((V,CV)))) ]))@\\
\mbox{}\verb@@{\NWsep}
\end{list}
\vspace{-1ex}
\footnotesize\addtolength{\baselineskip}{-1ex}
\begin{list}{}{\setlength{\itemsep}{-\parsep}\setlength{\itemindent}{-\leftmargin}}
\item \NWtxtMacroRefIn\ \NWlink{nuweb26b}{26b}.
\end{list}
\end{minipage}\\[4ex]
\end{flushleft}
%-------------------------------------------------------------------------------







\bibliographystyle{amsalpha}
\bibliography{larcc}

\end{document}
